%%%%%%%%%%%%%%%%%%%%%%%%%%%%%%%%%%%%%%%%%
% Draft
% 
%
%%%%%%%%%%%%%%%%%%%%%%%%%%%%%%%%%%%%%%%%%
\documentclass[a4paper,10pt]{article}  %scrartcl

\input{C:/Users/falves/Dropbox/TexFolder/preamble.tex} 
% \input{/Users/felipealves/Dropbox/TexFolder/preamble.tex}

\title{ Example State-Dependent Pricing 2 \vspace{-5em}} % use 5
% {
%         \vspace{-0in}  
%         \usefont{OT1}{bch}{b}{n}
%         \normalfont \normalsize \
%         \horrule{0.5pt} \[0.0cm]
%         \huge Referee report on ``Learning, Confidence, and Business Cycles''  \[-0.5cm]
%         \horrule{2pt} \[-0.5cm]
% }
% \author{
%         \normalfont \normalsize Felipe Alves \\[-3pt]       \normalsize
%         \today
% }
\date{ \vspace{-0em} }

% *************************************************************************************************************
% ************************************************                ********************************************* 
% *************************************************************************************************************

\begin{document}
\maketitle

%**************************************************************************************************************************
%%%%%%% INTODUCTION
%--------------------------------------------------------------------------------------------------------------------------
\section{Intoduction} % (fold)
\label{sec:intoduction}
%%%
\begin{itemize}
   
   %%
   \item Some comment

\end{itemize}
%%%
% SECTION INTODUCTION (END)
%..........................................................................................................................
%**************************************************************************************************************
%%%%%%% MODEL
%--------------------------------------------------------------------------------------------------------------
\section{Model} % (fold)
\label{sec:model}
%%%
\begin{itemize}
   %% 
   \item Production function
   %%%
   \begin{equation}
      \label{eq:model_eq1}
      y_t(h) = Z_t a_t(h) \ell_t(h)
   \end{equation}
   %%%

   %% 
   \item Profits
   
   %% 
   \item Each firm chooses prices $ \{\} $ in order to maximize its market value
   %%%
   \begin{equation}
      \label{eq:model_eq}
      \mathbb{E}_t \sum_{\tau=0}^{\infty} D_{t,t+\tau} \Pi_{t+\tau} (h)
   \end{equation}
   %%%
   where nominal profits in period $ t $ are given by
   %%%
   \begin{equation}
      \label{eq:model_eq3}
      \Pi_t(h) = p_t(h) y_t(h) - W_t \ell_t(h) - \kappa_t(h) \mathbbm{1}\{p_t(h) \ne p_{t-1}(h)\}
   \end{equation}
   %%%
   %%
   \item Value function for the firm
   \begin{multline}
      V_t \left( a_t(h), \frac{ p_{t-1}(h) }{ P_t }, \xi_t; \ \cdot \ \right) = 
      \max_{p} 
      \Bigg\{
         \Pi^R \left( a_t(h), \frac{ p_{t-1}(h) }{ P_t }, \cdot \right) 
         -\mathbbm{1} \{p \ne p_{t-1}(h ) \} \xi + \\
         %%
         + \mathbb{E}_t \bigg[ D^R_{t,t+1} V_{t+1} \left( a_{t+1}(h), \frac{ p }{ P_{t+1} }, \xi_{t+1}, \cdot \right) \bigg] \Bigg\}
   \end{multline}

   %% 
   \item Rewriting the problem
   \[
      v(a, \tilde{p}; \ \cdot \ ) = \int_{\xi} \max \Big\{ V^A(a,\tilde{p} ) - \xi w(\cdot) , \ V^N(a,\tilde{p} )  \Big\} dH(\xi )
   \]
   where
   %%%
   \begin{equation}
      \label{eq:model_eq4}
      %%%
      \begin{split}
      V^A(a, \tilde{p}_{-1}, \ \cdot \ ) & = \max_{\tilde{p}} 
         \Bigg\{ 
            \Pi^R \left( a, \tilde{p}, \cdot \right) + 
            \mathbb{E} \bigg[ D^R(\cdot,\cdot) v\Big( a,\tilde{p} \pi_{t+1}^{-1} \Big) \bigg]
         \Bigg\} \\
      %%
      V^N(a, \tilde{p}_{-1}, \ \cdot \ ) & = 
          \Pi^R \left( a, \tilde{p}_{-1}, \cdot \right) + \mathbb{E}
          \Big[ 
               D^R(\cdot,\cdot) v\Big( a', \tilde{p}_{-1}\pi_{t+1}^{-1} \Big)
          \Big]
      \end{split}
   \end{equation}
   %%%
   The firm will choose to pay the fixed cost iff $ V^A -\xi \ge V^N $. Hence,
   there is a unique threshold which makes the firm indifferent between these
   two options
   \[
      \tilde{\xi}(a, \tilde{p} ; ) = \frac{V^A(a, \tilde{p} ) - V^N( a, \tilde{p})}{w}
   \]

   %% 
   \item The firm value function V , adjusted by the marginal utility of the representative households, is therefore given by
\end{itemize}
%%%

%%%%%%% Household  
%--------------------------------------------------------------------------------------------------------------------------
\subsection{Household} % (fold)
\label{sub:household}



% Subsection household (end)
%..........................................................................................................................

%%%%%%% Equilibrium  
%--------------------------------------------------------------------------------------------------------------------------
\subsection{Equilibrium} % (fold)
\label{sub:equilibrium}

\begin{equil}
   A recursive competitive equilibrium is a set of value functions $ \Big\{ v, V^A, V^{N} \Big\} $,
   policies $ \{ \tilde{p}, \xi \} $ for the firm and household $ \Big\{ C( ), N( ) \Big\} $, 
   and wage $ w $ such that
   %%%
   \begin{enumerate}
      
      %%
      \item Firm optimization \\
      Taking $ w,Y $ as given the value function solves the Bellman equation and
      the $ \{ \tilde{p}, \xi \} $ are the associated policies

      %% 
      \item Household optimization \\
      \[
         R_t \mathbb{E} \bigg\{ \beta \frac{ u_c( C_{t+1} ) }{ u_c(C_t) } \frac{ P_{t} }{ P_{t+1} } \bigg\} = 1 
         \qquad
         N^{1/\varphi} = \frac{1}{\chi} C^{-\sigma}  w
      \]

      %% 
      \item Market clearing\\
      %% 
      $\bullet $ Labor market
      %%%
      \begin{equation}
         \label{eq:labor_market}
         \left( \frac{1}{\chi} C^{-\sigma}  w \right)^{\varphi} = 
         \int 
         \left[ 
            \frac{ \tilde{p}( a,\tilde{p}_{-1} )^{-\epsilon}Y }{ a }  + \left( \int^{\xi( )} \zeta dH( \zeta) \right) w 
         \right]d\mu
      \end{equation}
      %%%
      %% 
      $\bullet $ Goods market
      %%%
      \begin{equation}
         \label{eq:good_market}
         C_t = Y_t = \left( \int y(h)^{ \frac{\epsilon-1}{\epsilon} } d\mu \right)^{ \frac{\epsilon}{\epsilon-1} }
      \end{equation}
      %%%
      where $ y( a,\tilde{p}_{-1} ) = \Big( \tilde{p}(a,\tilde{p}_{-1}) \Big)^{-\epsilon} Y  $
      %% 
      \item Law of motion Distribution
   \end{enumerate}
   %%%
\end{equil}
% Subsection equilibrium (end)
%..........................................................................................................................


%%%%%%% Computation  
%--------------------------------------------------------------------------------------------------------------------------
\subsection{Computation} % (fold)
\label{sub:computation}

Compute Steady State
%%%
\begin{enumerate}
   
   %%
   \item Guess a value for the wage $ w^* $
   %% 
   \item Given $ w^* $ compute the firm's value function by iterating on Bellman equation. 
   Note that $ Y $ can be suppressed from the stationary Bellman because it is a multiplicative
   constant. 
   %% 
   \item Using firm's decision rules, compute the invariant distribution
   %% 
   \item Compute aggregate supply using the invariant distribution 
   \[
      \frac{C}{Y} = \left( \int  \tilde{p}(a,\tilde{p}_{-1})^{1 -\epsilon} d\mu \right)^{ \frac{\epsilon}{\epsilon-1} }
   \]
   If $ <1 $ increase $ w $ otherwise decrease $ w $ ( {\em \color{blue} Check on code} )
\end{enumerate}
%%%

% Subsection computation (end)
%..........................................................................................................................
% SECTION MODEL (END)
%..............................................................................................................

%%%%%%%%%%%%%%%%%%%%%%%%%%%%%%%%%%%%%%%%%%%%%%%%%%%%%%%%%%%%%%%%%%%%%%%%%%%%%%%%%%%%%%%%%%%%%%%%%%%%%%%%%%%%%%%
%%% End document


% ============================================================================================================
%------------------------------------------------ REFERENCE ------------------------------------------------
% ============================================================================================================
\clearpage
\newpage
%%%%% MAC
% \bibliographystyle{/Users/felipealves/Dropbox/TexFolder/plainnat2}
% \bibliography{/Users/felipealves/Dropbox/TexFolder/sample2}
%%%%% PC
\bibliographystyle{C:/Users/falves/Dropbox/TexFolder/plainnat2}
% \bibliography{C:/Users/falves/Dropbox/TexFolder/}

\end{document}  
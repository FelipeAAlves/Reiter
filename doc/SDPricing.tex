%%%%%%%%%%%%%%%%%%%%%%%%%%%%%%%%%%%%%%%%%
% Draft
% 
%
%%%%%%%%%%%%%%%%%%%%%%%%%%%%%%%%%%%%%%%%%
\documentclass[a4paper,10pt]{article}  %scrartcl
\usepackage{preview}

\input{C:/Users/falves/Dropbox/TexFolder/preamble.tex} 
% \input{/Users/felipealves/Dropbox/TexFolder/preamble.tex}

%%% EXTRA Packages
\usepackage{empheq}
\newcommand{\minus}[1]{{#1}^{-}}
\newcommand{\plus}[1]{{#1}^{+}}
\title{ Example State-Dependent Pricing \vspace{-1.25em}} % use 5
% {
%         \vspace{-0in}  
%         \usefont{OT1}{bch}{b}{n}
%         \normalfont \normalsize \
%         \horrule{0.5pt} \[0.0cm]
%         \huge Referee report on ``Learning, Confidence, and Business Cycles''  \[-0.5cm]
%         \horrule{2pt} \[-0.5cm]
% }
\author{
        \normalfont \normalsize Felipe Alves \\[-3pt]       \normalsize
        \today
}
\date{ \vspace{-3em} }

% *************************************************************************************************************
% ************************************************                ********************************************* 
% *************************************************************************************************************

\begin{document}
\maketitle

%**************************************************************************************************************************
%%%%%%% INTODUCTION
%--------------------------------------------------------------------------------------------------------------------------
\section{Intoduction} % (fold)
\label{sec:intoduction}
%%%
\begin{itemize}
   
   %%
   \item Some comment

\end{itemize}
 %%%
% SECTION INTODUCTION (END)
%..........................................................................................................................
%**************************************************************************************************************
%%%%%%% MODEL
%--------------------------------------------------------------------------------------------------------------
\section{Model} % (fold)
\label{sec:model}

%%%
\begin{itemize}
   %% 
   \item Production function
   %%%
   \begin{equation}
      \label{eq:model_eq1}
      y_t(h) = Z_t a_t(h) \ell_t(h)
   \end{equation}
   %%%

   %% 
   \item Each firm $h$ chooses prices $ \{p_t\}_{t} $ in order to maximize its market value
   %%%
   \begin{equation}
      \label{eq:model_eq}
      \mathbb{E}_t \sum_{\tau=0}^{\infty} D_{t,t+\tau} \Pi_{t+\tau} (h)
   \end{equation}
   %%%
   where $D_{t,t+\tau}$ is the nominal stochastic discount factor of the agent and $\Pi_{t+\tau} (h)  $
   are the nominal profits in period $ t $ given by
   %%%
   \begin{equation}
      \label{eq:model_eq3}
      \Pi_t(h) = p_t(h) y_t(h) - W_t \ell_t(h) - \xi_t(h) W_t \mathbbm{1}\{p_t(h) \ne p_{t-1}(h)\}
   \end{equation}
   %%%
   %%
   
   \item Value function for the firm
   \begin{multline}
      V_t \left( a_t(h), \frac{ p_{t-1}(h) }{ P_t }, \xi_t; \ \cdot \ \right) = 
      \max_{p} 
      \Bigg\{
         \Pi^R \left( a_t(h), \frac{ p }{ P_t }, \cdot \right) 
         -\mathbbm{1} \{p \ne p_{t-1}(h ) \} \xi w + \\
         %%
         + \mathbb{E}_t \bigg[ D^R_{t,t+1} V_{t+1} \left( a_{t+1}(h), \frac{ p }{ P_{t+1} }, \xi_{t+1}, \cdot \right) \bigg] \Bigg\}
   \end{multline}
   % \begin{multline}
   %    V_t \left( a, \frac{ p_{-1}(h) }{ P }, \xi; \ \cdot \ \right) = 
   %    \max_{p} 
   %    \Bigg\{
   %       \Pi^R \left( a, \frac{ p_{-1} }{ P }, \cdot \right) 
   %       -\mathbbm{1} \{p \ne p_{-1} \} \xi + \\
   %       %%
   %       + \mathbb{E}_t \bigg[ D^R_{t,t+1} V_{t+1} \left( a_{t+1}(h), \frac{ p }{ P_{t+1} }, \xi_{t+1}, \cdot \right) \bigg] \Bigg\}
   % \end{multline}


   %% 
   \item Rewriting the problem
   \[
      v(a, x; \ \cdot \ ) = \int_{\xi} \max \Big\{ V^A(a,x; \ \cdot \ ) - \xi w(\cdot) , \ V^N(a,x ; \ \cdot \ )  \Big\} dH(\xi )
   \]
   where
   %%%
   \begin{equation}
      \label{eq:model_eq4}
      %%%
      \begin{split}
      V^A(a, x; \ \cdot \ ) & = \max_{\tilde{p}} 
         \Bigg\{ 
            \Pi^R \left( a, \tilde{p}, \cdot \right) + 
            \mathbb{E} \bigg[ D^R(\cdot,\cdot) v\Big( a,\tilde{p} \pi_{t+1}^{-1} ; \ \cdot \ \Big) \bigg]
         \Bigg\} \\
      %%
      V^N(a, x; \ \cdot \ ) & = 
          \Pi^R \left( a, x, \cdot \right) + \mathbb{E}
          \Big[ 
               D^R(\cdot,\cdot) v\Big( a', x\pi_{t+1}^{-1} ; \ \cdot \  \Big)
          \Big]
      \end{split}
   \end{equation}
   %%%
   The firm will choose to pay the fixed cost iff $ V^A -\xi \ge V^N $. Hence, for 
   each individual state $ a, x $ there is a unique threshold which makes the firm 
   indifferent between these two options
   \[
      \tilde{\xi}(a, \tilde{p} ; ) = \frac{V^A(a, \tilde{p} ) - V^N( a, \tilde{p})}{w}
   \]

   %% 
   \item The firm value function $V$ is therefore given by
   \begin{equation}
      v(a, x; \ \cdot \ ) = 
         \int_0^{ \tilde{\xi}(a,x ) } 
         \Big[ 
            V^A(a, x ; \ \cdot \ ) - \xi w(\cdot) 
         \Big]d\xi  
         + \Big[  1 - H\Big( \tilde{\xi}(a, x ; ) \Big) \Big] V^N(a,x ; \ \cdot \ )
   \end{equation}
\end{itemize}
%%%

%%%%%%% Household  
%--------------------------------------------------------------------------------------------------------------------------
\subsection{Household} % (fold)
\label{sub:household}

There is a representative household with preferences represented by the utility function
%%%
\begin{equation}
   \label{eq:utility}
   E_0 \sum_{t=1}^{\infty} \beta^t \bigg[ \frac{ C_t^{1-\sigma} -1}{ 1-\sigma } - \chi \frac{ N_t^{1+ 1/\varphi} }{ 1+ 1/\varphi } \bigg]   
\end{equation}
%%%
The total time endowment per period is normalized to 1, so . 
The household owns the firms and markets are complete. 
% Subsection household (end)
%..........................................................................................................................

%%%%%%% Equilibrium  
%--------------------------------------------------------------------------------------------------------------------------
\subsection{Equilibrium} % (fold)
\label{sub:equilibrium}

\begin{equil}
   A recursive competitive equilibrium is a set of value functions $ \Big\{ v, V^A, V^{N} \Big\} $,
   policies $ \{ \tilde{p}, \xi \} $ for the firm and household $ \Big\{ C( ), N( ) \Big\} $, 
   and wage $ w(\cdot) $ such that
   %%%
   \begin{enumerate}
      
      %%
      \item Firm optimization \\
      Taking $ w(),Y() $ as given the value function solves the Bellman equation in XX and
      the $ \{ \tilde{p}, \xi \} $ are the associated policies

      %% 
      \item Household optimization \\
      \[
         R_t \mathbb{E} \bigg\{ \beta \frac{ u_c( C_{t+1} ) }{ u_c(C_t) } \frac{ P_{t} }{ P_{t+1} } \bigg\} = 1 , 
         \qquad
         N^{1/\varphi} = \frac{1}{\chi} C^{-\sigma}  w( \cdot )
      \]

      %% 
      \item Market clearing \\ \ \\
      
      %%
      $\bullet $ Bonds market
      
      \hspace{1em} Simply requires that $ B_t = 0 $ {\color{RubineRed} Include this?}

      %% 
      $\bullet $ Labor market
      %%%
      \begin{equation}
         \label{eq:labor_market}
         %%%
         \begin{split}
            % \left( \frac{1}{\chi} C^{-\sigma}  w \right)^{\varphi} & = 
            N & = 
            \int \int_0^{ \bar{\xi} }
            \Bigg[ 
               \ell \Big(a, \tilde{p}(a,x,\xi; \bsy{s} ) ; \bsy{s}\Big) + 
               \mathbbm{1} \Big\{ \tilde{p}(a,x,\xi; \bsy{s} ) \ne x \Big\} \xi
            \Bigg]
            dH( \xi ) \ d\widetilde{\Psi} \\           
            & = \int 
            \Bigg[ 
                  H\Big( \tilde{\xi}( a,x;\bsy{s} ) \Big) \ell\Big(a, \tilde{p}^a(a; \bsy{s} ) ; \bsy{s} \Big) + \\
            & \qquad \qquad
                  \left( 1- H\Big( \tilde{\xi}( a,x;\bsy{s} ) \Big) \right) \ell\Big(a, x ; \bsy{s} \Big) + 
                  \int_0^{ \tilde{\xi}(a,x; \bsy{s} ) } \xi dH(\xi)
            \Bigg] d\widetilde{\Psi}
         \end{split}
      \end{equation}
      %%%
      where $ \ell(a,\tilde{p}; \bsy{s} ) = \frac{ \tilde{p}^{-\epsilon} Y }{ Z a } $ \\
      %% 
      $\bullet $ Goods market
      %%%
      \begin{equation}
         \label{eq:goods_market}
         %%%
         \begin{split}
            C_t = Y_t & = \left( \int_0^1 y(h)^{ \frac{\epsilon-1}{\epsilon} } dh \right)^{ \frac{\epsilon}{\epsilon-1} } \\
                  1 & = \int \int_0^{ \bar{\xi} } \tilde{p} (a, x,\xi; \bsy{s} )^{1-\epsilon} dH(\xi) d\widetilde{\Psi}
         \end{split}
      \end{equation}
      %%%
      where $ y( a,x ) = \Big( \tilde{p}(a,x) \Big)^{-\epsilon} Y  $
      %% 
      \item Law of motion Distribution
   \end{enumerate}
   %%%
\end{equil}
% Subsection equilibrium (end)
%............................................................................................................

% SECTION MODEL (END)
%............................................................................................................

%************************************************************************************************************
%%%%%%% METHOD
%------------------------------------------------------------------------------------------------------------
\section{Method} % (fold)
\label{sec:method}

%%%%%%% Finite Dimensional Approximation  
%--------------------------------------------------------------------------------------------------------------------------
\subsection{Finite Dimensional Approximation} % (fold)
\label{sub:finite_dimensional_approximation}


\textbf{Firm's Value Function}
%%%
\begin{itemize}[label=\raisebox{0.25ex}{\tiny $\bullet$ }]
   
   %%
   \item Value functions are differently approximated in steady state and on the perturbation step 
   %%%
   \begin{equation}
      \label{eq:approx_value_function}
      v\big( a, x; \bsy{s} \big) \approx \sum_{j=1}^{n_a} \sum_{i=1}^{n_x} \theta_{j,i}( \bsy{s} ) \psi_{j,i} (a,x)
   \end{equation}
   %%%
   %%
   \item With a particular approximation of the value function, we solve for the coefficients at steady state 
   using collocation which forces the equation to hold exactly on a set of grid points $ \big\{a_j, x_i \big\}_{j=1,n_A \ i =1,n} $
   %%%
   \begin{align*}
      %%
      % \label{eq:_eq1}
      v \big( a_j, x_i; \bsy{\theta}^* \big) & = 
      H \Big( \tilde{\xi}(a_j,x_i) \Big) 
         \Bigg\{ 
               \Pi^R\Big( a_j, \tilde{p}^a (a_j, x_i) ; \ \cdot \ \Big) + \beta\sum_{a_{j'}\in \mathcal{A} } \Pi(a_j,a_{j'})
               v \left( a_{j'},\frac{\tilde{p}^a (a_j, x_i)}{\pi^*} ; \bsy{\theta}^* \right) 
         \Bigg\} \  + \\
      %%
      % \label{eq:_eq2}
      & \qquad \int_0^{ \tilde{\xi}(a_j,x_i) } \xi dH(\xi) \quad + \\
      & \qquad \left( 1- H\Big( \tilde{\xi}(a_j,x_i) \Big) \right) 
         \Bigg\{
               \Pi^R\Big( a, x_i ; \ \cdot \ \Big) + \beta\sum_{a'\in \mathcal{A} } \Pi(a,a')
               v \bigg( a',\frac{x_i}{\pi^*} ; \bsy{\theta}^* \bigg) 
         \Bigg\}
   \end{align*}
   %%%
   where the decision rules are computed from
   %%%
   \begin{align}
      %%
      \label{eq:policies_eq1}
      & \tilde{\xi}(a_j, x_i) = \frac{ V^A(a_j, x_i ; \bsy{\theta}^*) - V^N(a_j, x_i ; \bsy{\theta}^*) }{ w^* } \\
      %%
      \label{eq:policies_eq2}
      & 0 = Y^* \tilde{p}^{a}( a_j )  - \epsilon Y^* \left( \tilde{p}^a( a_j ) - \frac{ w^* }{a_j} \right) + 
      \beta \Big( \tilde{p}^a( a_j ) \Big)^{\epsilon} 
      \Bigg[ 
            \sum_{a\in \mathcal{A} } \Pi (a,a') \frac{ \partial v \big( a', \tilde{p}^a( a_j )/\pi^* ; \bsy{\theta}^* \big) }{ \partial x' } 
      \Bigg]
   \end{align}
   %%%
   %%
   \item Note that the conditional expectation of the future value function has been broken into two components:
    the expectation with respect to idiosyncratic shocks is taken explicitly by integration while the expectation 
    with respect to the aggregate shocks is denoted by the expectation operator. 

    $ \theta^\prime $
\end{itemize}
%%%

\textbf{Distribution}

Two relevant distributions
%%%
\begin{itemize}[label=\raisebox{0.25ex}{\tiny $\bullet$ }]
   
   %%
   \item $ \widetilde{\Psi}(a,x) $ : distribution over beginning of period real prices
   %%
   \item $ \Psi(a, \tilde{p}) $ : distribution over effective real prices (production relevant) and idio shocks $
   \mathbf{t} $
\end{itemize}
%%%
The $\widetilde{\Psi}$ distribution dynamics involves 3 different steps
%%%
\begin{enumerate}
   %%
   \item decision of price adjustment adjustment 
   %%
   \item exogenous transition
   %%
   \item deflation by inflation between $ t\mapsto t+1 $
\end{enumerate}
%%%
The first option comes from \citet{reiter} and involves approximating both distributions by a finite number of mass
points on a predefined grid on $ \mathcal{X} \defeq \big\{ x_i \big\}_{i=1}^{N_X} $ and productivity $\mathcal{A}$.
%%
Let $ \widetilde{\Psi}( a_j, \tilde{p}_{-1} ) $ denote the fraction of firms at the beginning of the period with
productivity level $ a_j $ and last period relative price of $ \tilde{p}_{-1}$. The evolution of this distribution
involves the three steps discussed above. For any pairs $\Big\{ ( a_{j}, \tilde{p}_{\text{-} 1}^i ) , ( a_{j}, x_{i'}
)\Big\}  \subset \mathcal{A}\times \mathcal{X} $, the probability of moving from the first to the second is
%%%
\begin{equation}
   \label{eq:transition_inflation}
   prob( \qquad ; \Pi) = 
   \begin{cases}
      \frac{ \Pi^{-1} \tilde{p}_{\text{-} 1}^i - x_{i'-1} }{ x_{i'} - x_{i'-1}}   & \text{if } x_{i'}=\min \big\{ x \in
      \mathcal{X} : x \ge \Pi^{-1} \tilde{p}_{\text{-} 1}^i \big\} \\
      \frac{ x_{i'+1} - \Pi^{-1} \tilde{p}_{\text{-} 1}^i }{ x_{i'+1} - x_{i'}}   & \text{if } x_{i'}=\max \big\{ x \in
      \mathcal{X} : x < \Pi^{-1} \tilde{p}_{\text{-} 1}^i \big\}   \\ 0 & \text{ow}
   \end{cases}
\end{equation}
%%%
Then, for any pair $ ( a_{j}, \tilde{p}^i ) \in \mathcal{A}\times \mathcal{X} $, next period ${\Psi}'$
%%
\begin{equation}
   \Psi^{\prime} \Big( a_{j}, \tilde{p}^{i'} \Big) =  
         \omega_{j, i'} \sum_{ i=1 }^{N_X} \xi_{j,i} \times \widetilde{\Psi}(a_j, x_i) + 
         (1-\xi_{ j,{i'} }) \times \widetilde{\Psi}(a_j, x_{i'} )
\end{equation}
where $\omega_{j, i'}$
\begin{equation*}
   \label{eq:weight}
   \omega_{j,i'} = \begin{cases}
   \frac{ \tilde{p}_j - \tilde{p}^{i'-1} }{ \tilde{p}^{i'} - \tilde{p}^{i'-1} } & \text{ if } \tilde{p}_j \in \Big[
   \tilde{p}^{i'-1},\tilde{p}^{i'} \Big] \\ \ & \ \\
   \frac{ \tilde{p}^{i'+1} - \tilde{p}_j }{ \tilde{p}^{i'+1} - \tilde{p}^{i'} } & \text{ if } \tilde{p}_j \in \Big[
   \tilde{p}^{i'},\tilde{p}^{i'+1} \Big] \\ \ & \ \\ 0 & \text{o.w.}
   \end{cases}
\end{equation*}

\newpage
%%%%%%%%%%%%%%%%%%%
\textbf{Finite-Dimensional system}
\bigskip
%%%
% \begin{equation}
   % \label{eq:}
\footnotesize
   \begin{empheq} [left=\empheqlbrace]{gather*} 
      %% HOUSEHOLD
      \beta \left( \frac{Y'}{Y} \right)^{-\sigma} \frac{1}{\Pi'}R -1 \\
      N^{1/\varphi}  - \frac{1}{\chi} Y^{-\sigma} w \\ \ \\
      % -----------------------------------------------------------------------------------------------------   
      %% Market Clearing conditions
      N - \int 
            \Bigg[ 
                  H\Big( \tilde{\bsy{\xi}}_{j,i} \Big) 
                     \frac{ \big( \tilde{p}^a(a) \big)^{-\epsilon} Y }{ Za } +
                  \left( 1- H\Big( \tilde{\bsy{\xi}}_{j,i} \Big) \right) 
                     \frac{ x^{-\epsilon} Y }{ Za } +
                  \int_0^{ \tilde{\bsy{\xi}}_{j,i} } \xi dH(\xi)
            \Bigg] d\widetilde{\Psi}(\Pi) \\
      1 - \int 
            \Bigg[ 
                  H\Big( \tilde{\bsy{\xi}}_{j,i} \Big) \big( \tilde{p}^a ( a ) \big)^{1-\epsilon} + 
                  \left( 1- H\Big( \tilde{\bsy{\xi}}_{j,i} \Big) \right) x^{1-\epsilon}
            \Bigg] d\widetilde{\Psi}(\Pi)\\
            % 
      1 - \int \tilde{p}^{1-\epsilon} d\Psi'(\Pi) \\ \ \\
      % -----------------------------------------------------------------------------------------------------   
      %%% Dynamics of the Distribution
      \text{Distribution Dynamics}      \\ \ \\
      %% Exogenous
      z'  - \rho_z z - \sigma_{z} \omega_z' \\ \ \\
      % ( \tau^k )'  - \rho_k \tau^k - \sigma_{k} \omega_k' \\ \ \\
      %
      % -----------------------------------------------------------------------------------------------------   
      %% Value Function
      \bsy{V}_{j,i} - H\Big( \bsy{\tilde{\xi}}_{j,i} \Big) 
         \Bigg\{
                  \Pi^R \Big( a_j, \bsy{p}^a_{j} ; \ \cdot \ \Big) + 
                  \beta \left( \frac{Y'}{Y} \right)^{-\sigma}  \sum_{j'} \Pi[ a_j,a_{j'} ] 
                  \Big( \bsy{V}',v^* \Big) \left( a_{j'}, \frac{\bsy{p}^a_{j}}{ \Pi' } \right)
         \Bigg\} \\
         + \int_0^{ \tilde{\bsy{\xi}}_{j,i} } \xi dH(\xi)
         - \left( 1- H\Big(  \tilde{\bsy{\xi}}_{j,i}  \Big) \right) 
         \Bigg\{
               \Pi^R\Big( a, x_i ; \ \cdot \ \Big) + 
               \beta\sum_{j'} \Pi(a_j,a_{j'})
                  \Big( \bsy{V}',v^* \Big) \bigg( a_{j'}, \frac{x_i}{ \Pi' } \bigg)
         \Bigg\} \\
      % -----------------------------------------------------------------------------------------------------   
      %% FOC
         \frac{ \partial \Pi^R}{ \partial \tilde{p} } \Big( a_j, \bsy{p}^a_j ; \ \cdot \ \Big) + \beta \left(
         \frac{Y'}{Y} \right)^{-\sigma}  \sum_{j'} \Pi[ a_j,a_{j'} ]
                  \frac{ \partial \Big( \bsy{V}',v^* \Big) }{ \partial x' } \left( a_{j'}, \frac{\bsy{p}^a_{j}}{ \Pi' }
                  \right) \frac{1}{\Pi'}
   \end{empheq}
\normalsize
% \end{equation}
With all these approximations, the recursive equilibrium becomes computable. 

function $ f $ that satisfies
%%%
\begin{equation}
   \label{eq:equil_conditions}
   \mathbb{E} \Big[ f \big( \mathbf{y}',\mathbf{y}, \mathbf{x}', \mathbf{x} \big) \Big] = 0
\end{equation}
%%%
where $ \textbf{y} = \big( Y,N,R,\Pi,w, \bsy{V}, \bsy{p}^a, \bsy{\xi} \big) $ are the control variables, 
$ \mathbf{x} = \big(\Psi, Z \big) $ are the (endogenous and exogenous) state variables.

This puts the model in the canonical form presented in \citet{schmitt-uribe}. 

% which we can recast on Sims form
%%%
% \begin{equation}
%    \label{eq:sims_format}
%    \Gamma_0 X_t = \Gamma_1 X_{t-1} + \Psi \epsilon_t + \Pi \eta_t
% \end{equation}
% %%%
% by defining $ X_t = \Big[ \mathbf{y}_t,\mathbf{x}_t \Big] $ and introducing endogenous forecast errors $ \eta_t $
% for the control variables $ E_t \Big[ \mathbf{y}_{t+1} \Big] = \mathbf{y}_{t+1} - \eta_{t+1} $. The expectation over states
% next period is dealt by recognizing that uncertainty there is only related to the evolution of the exogenous process, which
% are explicitly in \eqref{eq:sims_format} through $\epsilon_t$.
% %
% Doing that, we are able to represent \eqref{eq:linearized_equil_conditions} in the form \eqref{eq:sims_format}
% with coefficients
% \[
%    \Gamma_0 = [ -f_{\mathbf{y}'} \ -f_{\mathbf{x}'}], \quad \Gamma_1 = [ f_{\mathbf{y}} \ f_{\mathbf{x}}], \quad \Pi = -f_{\mathbf{y}'}
% \]
% where yt is an n×1 vector of endogenous variables, \EPSILON is a l×1 vector of exogenous,
% serially uncorrelated random disturbances, \eta_t is a k × 1 vector of expectation
% errors, satisfying Et−1[\etat ]=0 for all t. \Gamma_0 and \Gamma_1 are n×n coefficient matrices, while
%   \Psi  is n×l and \Pi
%  is n×k

%%%
% Subsection finite_dimensional_approximation (end)
%..........................................................................................................................

%%%%%%% Steady State  
%--------------------------------------------------------------------------------------------------------------------------
\subsection{Steady State} % (fold)
\label{sub:steady_state}

\textbf{OPTION 02}

%%%
\begin{enumerate}
   
   %%
   \item Guess a value a pair $ w^*,Y^* $
   %%
   \item Given $ \Big(w^*,Y^* \Big) $, compute the firm's value function.
   %%
   \item Using the firm's decision rules, compute the invariant distribution.
   %%
   \item Check the market-clearing conditions
      %%%
      \begin{align*}
         %%
         % \label{eq:_eq1}
         1 & = \int \tilde{p}^{1-\epsilon} \Psi\big(da, d\tilde{p} \big) \\
         %%
         % \label{eq:_eq2}
         N^* & = 
         \int \ell \big( a, \tilde{p} \big) \Psi \big(da,d\tilde{p} \big) + 
         \int \left( \int^{\tilde{\xi}( a, x) } \zeta dH( \zeta) \right) \widetilde{\Psi} \big( da,dx \big)
      \end{align*}
      %%%
\end{enumerate}
%%%
% Subsection steady_state (end)
%............................................................................................................


% SECTION METHOD (END)
%..................................................................................................


\newpage
%**************************************************************************************************
%%%%%%% LINEAR_RATIONAL_EXPECTATIONAL_DIFFERENCE_EQUATIONS
%--------------------------------------------------------------------------------------------------
\section{Linear Rational Expectational Difference Equations} % (fold)
\label{sec:linear_rational_expectational_difference_equations}

%%%%%%% Preliminaries  
%--------------------------------------------------------------------------------------------------------------------------
\subsection{Preliminaries} % (fold)
\label{sub:preliminaries}

\begin{definition}
   Let $ P\in \mathbb{C}\rightarrow \mathbb{C}^{n\times n} $ be a matrix-valued function of a complex variable (a matrix
pencil). Then the set of its generalized eigenvalues $ \lambda( P) $ is defined as
\[
   P(z) \defeq \big\{ z\in \mathbb{C}: \lvert P(z) \rvert = \mathbf{0} \big\}
\]
When $P(z)$ writes as $Az−B$, we denote this set as $\lambda(A,B)$. Then there exists
a vector $V$ such that $BV = \lambda AV$ .
\end{definition}

\begin{theorem}[The complex generalized Schur form]
Let $A$ and $B$ belong to $ \mathbb{C}^{n\times n} $ and be such that $ P(z) = Az - B$ is a regular matrix pencil. Then
there exist unitary $n\times n$ matrices of complex numbers $Q$ and $Z$ such that
%%%
\begin{enumerate}
   
   %%
   \item $ S = Q A Z $ is upper-triangular
   %%
   \item $ T = Q B Z $ is upper-triangular
   %%
   \item For each $ i $, $ s_{ii},t_{ii}$ are not both zero
   %%
   \item $ \lambda(A,B) = \left\{ \frac{t_{ii}}{s_ii} : \ s_{ii}\ne 0 \right\}$
   %%
   \item The pairs $ (s_{ii}, t_{ii} ) $ can be arranged in any order
\end{enumerate}
%%%

\end{theorem}

\begin{theorem}[\emph{Singular Value Decomposition}]
   Every $m\times n $ complex matrix $ A $ can be factored into a product of three matrices
   %%%
   \begin{equation}
      \label{eq:SVD}
      A = U S V'
   \end{equation}
   %%%
   called a \emph{singular value decomposition (SVD)} where
   %%%
   \begin{itemize}%[label=\raisebox{0.25ex}{\tiny $\bullet$ }]
      
      %%
      \item $ U _{m\times m}, V_{n\times n}$ are orthogonal matrices 
      %%
      \item $ S $ is a diagonal matrix with entries $ \sigma_1 \ge \sigma_2 \ge \ldots \ge \sigma_{\min\{n,m\}} \ge 0$. 
   \end{itemize}
   %%%
   The values $ \sigma_i $ are called the \textbf{singular values} of $ A $, while the column vectors $ u_j,v_i $ are
   the left/right \textbf{singular vectors}.
\end{theorem}

% Subsection preliminaries (end)
%..........................................................................................................................

%%%%%%% Klein  
%--------------------------------------------------------------------------------------------------------------------------
\subsection{Klein} % (fold)
\label{sub:klein}

Linearizing \eqref{eq:equil_conditions} around the steady-state yields a first-order linear expectational difference
equation system of the form
%%%
\begin{equation}
   \label{eq:linearized_equil_conditions}
   f_{ \mathbf{y}' } E_t \Big[ \underbrace{ ( \mathbf{y}_{t+1} -  \bar{\mathbf{y}}) }_{ \tilde{\mathbf{y}}_{t+1} } \Big] 
   + f_{ \mathbf{y} }   \tilde{\mathbf{y}}_t  
   + f_{ \mathbf{x}' }  E_t \Big[  \tilde{\mathbf{x}}_{t+1} \Big]
   + f_{ \mathbf{x} }   \tilde{\mathbf{x}}_{t}
   = 0
\end{equation}
%%%
which we can put on Klein's form letting $ A = - \big[f_{ \mathbf{x}' } \quad f_{ \mathbf{y}' } \big] $  and $ B
=\big[f_{ \mathbf{x} } \quad  f_{ \mathbf{y} } \big] $. Then the system can be written as
%%%
\begin{equation}
   \label{eq:klein}
   A E_t \left\{ 
   \begin{bmatrix}
      x_{t+1} \\ y_{t+1}
   \end{bmatrix}
   \right\} = 
   B 
   \begin{bmatrix}
      x_{t} \\ y_t 
   \end{bmatrix}
\end{equation}
%%%
Consider the generalized Schur decomposition of $ A, B $ 
%%%
\begin{equation}
   % \label{eq:}
   Q A Z = S \qquad Q B Z = T
\end{equation}
%%%
where $ A,B $ are upper triangular and $ Q,Z$ are orthonormal matrices. Let $ S $ and $ T $ be arranged in such a way
that the $ n_s $ stable eigenvalue come first. Partition the rows of $ Z $ conformably as 
%%%
\begin{equation*}
   % \label{eq:}
   Z = 
   \begin{bmatrix}
      Z_{11} & Z_{12} \\ Z_{21} & Z_{22}
   \end{bmatrix}
\end{equation*}
%%%
DEfine the auxiliary variables $ W_t $ as
%%%
\begin{equation}
   \label{eq:aux}
   w_t \defeq Z^{H} \big[ x_t' \ y_t' \big]' =
   \begin{bmatrix}
      s_t \\ u_t
   \end{bmatrix}
\end{equation}
%%%
where the transformed variable $ w_t $ is divided into $ n_s\times 1 $ stable and $ n_u \times 1 $ unstable components. 
Premultiply the system by $ Q $ to get 
%%%
\begin{equation}
   % \label{eq:}
   \begin{bmatrix}
      S_{11} & S_{12} \\ 0 & S_{22}
   \end{bmatrix}
   E_t \left\{
   \begin{bmatrix}
      s_{t+1} \\ u_{t+1}
   \end{bmatrix} \right\}
    = 
    \begin{bmatrix}
       T_{11} & T_{12} \\ 0 & T_{22}
    \end{bmatrix}
    \begin{bmatrix}
      s_{t} \\ u_{t}
   \end{bmatrix}
\end{equation}
%%%
Since the generalized eigenvalues of the matrix pencil $ S_{22}z - T_{22} $ are all unstable, the unique stable solution
for $ u_t $ is found by solving forward
%%%
\begin{equation}
   \label{eq:unstable_sol}
   u_t = T_{22}^{-1} S_{22} E_t\{ u_{t+1} \} = 0
\end{equation}
%%%
The first block then implies
%%%
\begin{equation}
   % \label{eq:}
   E_t\big\{ s_{t+1} \big\}  = S_{11}^{-1} T_{11} s_t
\end{equation}
%%%
Recalling the definition of $ w_t $
%%%
\begin{equation}
   % \label{eq:}
   x_{t+1} = 
      \begin{bmatrix}
         Z_{11} & Z_{12}
      \end{bmatrix} 
      \begin{bmatrix}
         s_{t+1} \\ u_{t}
      \end{bmatrix}
\end{equation}
%%%
we have
%%%
\begin{equation}
   \label{eq:}
   z_{11} \Big( s_{t+1} -  E_t\{ s_{t+1} \}\Big) = \eta \epsilon_{t+1}
\end{equation}
%%%
Taken together, these define the unique solution for $ s_t $ given $ s_0 $ and the exogenous process $ \epsilon $
%%%
\begin{equation}
   % \label{eq:}
   s_{t+1} = S_{11}^{-1} T_{11} s_t + Z_{11}^{-1} \eta \epsilon_{t+1}
\end{equation}
%%%
We can now use the relation $ \Big[x_t' \ y_t'\Big]' = Z w_t$ to find a recursive representation of the solution. Using
the restriction on the control 
%%%
\begin{align}
   %%
   % \label{eq:_eq1}
   Z^H_{21} x_t + Z^H_{22} y_t = 0 \Rightarrow y_t & = - \big( Z^H_{22} \big)^{-1} Z^{H}_{21} x_t 
   & \quad \quad & \notag \\
   %%
   % \label{eq:_eq2}
   & = \big( Z^H_{22} \big)^{-1} Z^H_{22} Z_{21} Z_{11}^{-1} x_t \notag \\
   & = Z_{21}Z_{11} x_t
\end{align}
%%%
As for $ x_t $, note that
%%%
\begin{align*}
   %%
   % \label{eq:_eq1}
   s_t & = Z^H_{11}x_t + Z^H_{12} y_t \\
       & = \Big( Z^H_{11} + Z^H_{12} Z_{21}Z_{11}^{-1} \Big) x_t \\
       & = \bigg( Z^H_{11} + \Big( Z_{11}^{-1} - Z^H_{11} \Big) \bigg) x_t \\
       & = Z_{11}^{-1} x_t
\end{align*}
%%%
Hence 
%%%
\begin{equation}
   % \label{eq:}
   x_{t+1} = Z_{11} S_{11}^{-1} T_{11} Z_{11}^{-1} x_t + \eta \epsilon_{t+1}
\end{equation}
%%%

% Subsection klein (end)
%..........................................................................................................................

%%%%%%% Sims  
%--------------------------------------------------------------------------------------------------------------------------
\subsection{Sims} % (fold)
\label{sub:sims}



Consider the following model 
%%%
\begin{equation}
   \label{eq:sims_format}
   \Gamma_0 y_t = \Gamma_1 y_{t-1} + \Psi \epsilon_t +\Pi \eta_t
\end{equation}
%%%
where $ y_t $ is a $ n\time 1 $ vector of endogenous variables, $ \Gamma_0,\Gamma_1 $ are $ n\times n $ coefficients
matrices , $ \epsilon_t $ is a $ \ell\times 1 $ vector of exogenous random disturbances, $ \eta_t $ is an
$ k\times 1 $ vector of expectational errors satisfying $ E_t \big[ \eta_{t+1} \big] =0 $. 
Note that $ \eta_t $ terms are not given exogenously, instead they are determined as part of the solution. 

This method uses the notation that in which time arguments relate consistently to the information structure, meaning
that variables date $ t $ are always know at date $ t $.

%%%%%%% Special Case  
%------------------------------------------------------------------------------------------------------------
\subsubsection{Special Case} % (fold)
\label{sub:special_case}

Consider the special case of \eqref{eq:sims_format}
\[
   y_t = \Gamma y_{t-1} + \Psi \epsilon_t +\Pi \eta_t
\]
Assume that the matrix $ \Gamma $ can be diagonalized to
\[
   \Gamma = P \Lambda P^{-1}
\]
where $ P $ is the matrix of right-eigenvectors of $ \Gamma $, $ P^{-1} $ is the matrix of left-eigenvectors
and $ \Lambda $ is the diagonal matrix of eigenvalues. Multiplying the system by $ P^{-1} $ and defining $ w = P^{-1} y $
we arrive at
%%%
\begin{equation}
   \label{eq:diag_system}
   w_t = \Lambda w_{t-1} + P^{-1} \Big( \Psi \epsilon_t + \Pi \eta_t \Big)
\end{equation}
%%%
Since $\Gamma$ is diagonal the system breaks into unrelated components
%%%
\begin{equation}
   \label{eq:single_eq}
   w_{j,t} = \lambda_j w_{j,t-1} + \tilde{P}_{j\cdot} \Big( \Psi \epsilon_t + \Pi \eta_t \Big)
\end{equation}
%%%
If the disturbance term zero and $ \lambda_j \ne 1$, the model has a deterministic steady-state solution 
%%%
\begin{equation}
   \label{eq:restriction}
   w_{j,t} = 0 
\end{equation}
%%%
If $ \lvert \lambda_j \rvert > 1$, then $ E_t\big[ w_{j,t+\tau} \big] $ diverges as $ \tau\rightarrow \infty $ for any
solution other then $ w_{j,t} = 0 $. If we are looking for a \emph{stationary equilibrium}, every one of 
the variables $ w_j $ corresponding to $ \lvert \lambda_j \rvert > 1 $ and to $ P_{j,:}\ne 0 $ must be set to its steady-state value. 
If \eqref{eq:restriction} holds for all $ t $ then \eqref{eq:single_eq} implies
%%%
\begin{equation}
   \label{eq:eta_rest_01}
   \tilde{P}_{j\cdot} \Big( \Psi \epsilon_t + \Pi \eta_t \Big) = 0 
\end{equation}
%%%
Collecting all the rows of $ P^{-1} $ for which \eqref{eq:eta_rest_01} holds into a single matrix $ \tilde{P}^U $, 
we can write
%%%
\begin{equation}
   \label{eq:eta_rest_02}
   \tilde{P}^U \Big( \Psi \epsilon + \Pi \eta \Big) = 0
\end{equation}
%%%
Existence problems arise if the endogenous shocks $ \eta $ cannot adjust to offset the exogenous disturbances $ \epsilon
$. This accounts for the usual notion that there are existence problems if the number of \emph{unstable roots} exceeds
the number of \emph{jump variables}. The precise condition here is that the columns of $ \tilde{P}^U \Pi $  span the
space spanned by the columns of $ \tilde{P}^U \Psi $, i.e.
%%%
\begin{equation}
   \label{eq:iexistence}
   \text{span} \Big( \tilde{P}^U \Psi \Big) \subset \text{span} \Big( \tilde{P}^U \Pi \Big)
\end{equation}
%%%
Multiple solutions may exist when \eqref{eq:iexistence} puts too few restrictions. For the solution to be unique, it
must be that \eqref{eq:iexistence} pins down not only the value of $ \tilde{P}^U \Pi \eta $ but also$ \tilde{P}^S \Pi
\eta $ which resumes the impact of expectational shocks on the stable block of the system \eqref{eq:diag_system}.
Formally, we require the row space of $  \tilde{P}^S \Pi  $ to be included into the row space of $ \tilde{P}^U \Pi $.
% %%%
% \begin{equation}
%    \label{eq:iuniqueness}
%    \text{span} \Big( \big( \tilde{P}^S \Pi \big)' \Big) \subset \text{span} \Big( \big( \tilde{P}^U \Pi \big) \Big)
% \end{equation}
% %%%
In that case, there exists $ \Phi $ such that
%%%
\begin{equation}
   \label{eq:row_space_transf}
   \tilde{P}^S \Pi = \Phi \tilde{P}^U \Pi
\end{equation}
%%%
We can write the solution by assembling the equations representing the stability conditions \eqref{eq:restriction}
together with the lines of \eqref{eq:diag_system} that determine $ w_s $ and use \eqref{eq:row_space_transf} to
eliminate the dependence over $ \eta $.
%%%
\begin{equation}
   \label{eq:system1_simul}
   \begin{bmatrix}
      w^S_{t} \\ w^U_{t}
   \end{bmatrix} = 
   \begin{bmatrix}
      \Lambda_S \\ \mathbf{0}
   \end{bmatrix} w^S_{t-1}
   \begin{bmatrix}
      I & -\Phi  \\ \mathbf{0} & \mathbf{0}
   \end{bmatrix}P^{-1} \Psi \epsilon_t
\end{equation}
%%%
Multiply by $ P $ to arrive in $ y $
%%%
\begin{equation}
   \label{eq:system2_simul}
   y_t = \underbrace{P_{:,S} \Lambda_S \tilde{P}^S}_{ \Theta_y } y_{t-1} + 
         \underbrace{\Big( P_{:,S} \tilde{P}^S - P_{:,S} \Phi \tilde{P}^U \Big) \Psi}
         _{\Theta_{\epsilon}} \epsilon_t
\end{equation}
%%%

% Subsection special_case (end)
%..................................................................................................

%%%%%%% General CAse  
%--------------------------------------------------------------------------------------------------
\subsubsection{General CAse} % (fold)
\label{sub:general_case}
\emph{Generalized Schur decomposition} there exist matrices $ Q,Z,T $ and $ S $
such that
%%%
\begin{align}
   %%
   % \label{eq:_eq1}
   & Q' S Z' = \Gamma_0, \qquad Q' T Z' = \Gamma_1 \\
   %%
   % \label{eq:_eq2}
   & QQ' = ZZ' = I_{n\times n}
\end{align}
%%%
Let us define $ w_t = Z'y_t $ and pre-multiply the system by $ Q $ in order to get
%%%
\begin{equation}
   \label{eq:QZsims_eq01}
   \begin{bmatrix}
      S_{11} & S_{12} \\ 0 & S_{22}
   \end{bmatrix}   
   \begin{bmatrix}
      w_{1,t} \\ w_{2,t}
   \end{bmatrix} = 
   \begin{bmatrix}
      T_{11} & T_{12} \\ 0 & T_{22}
   \end{bmatrix}
   \begin{bmatrix}
      w_{1,t-1} \\ w_{2,t-1}
   \end{bmatrix} +
   \begin{bmatrix}
      Q_1 \\ Q_2
   \end{bmatrix}
   \Big( \Psi \epsilon_t + \Pi \eta_t \Big) 
\end{equation}
%%%
Let's focus on the explosive part of the system 
\[
   S_{22} w_{2,t} = T_{22} w_{2,t-1} + Q_2\Big( \Psi \epsilon_t + \Pi \eta_t \Big) 
\]
While the diagonal elements of $ S_{22} $ can be null, $ T_{22} $ is necessarily full rank. Therefore, we can solve
forward for $ w_{2,t-1} $. Start by leading the equation by one period and writing it in terms of $ w_{2,t} $
\[
   w_{2,t} = M z_{2,t+1} - T_{22}^{-1} Q_2 \Big( \Psi \epsilon_{t+1} + \Pi \eta_{t+1} \Big) 
\]
where $ M \defeq T_{22}^{-1} S_{22} $. Recursive substitution of $ w_{2,t+1} $ leads us to 
\[
   w_{2,t} = - \sum_{i=1}^{\infty} M^{i-1} T_{22}^{-1} Q_2 \Big( \Psi \epsilon_{t+i} + \Pi \eta_{t+i} \Big)
\]
where we imposed $ \lim M^t w_{2,t} = 0 $ since we are searching for a non-explosive solution of the LRE model
\eqref{eq:sims_format}. Since $ y_{t} $ is known at time $t$, $ w_{2,t} = E_t \{w_{2,t} \}$ which implies
\[
   w_{2,t} = - \sum_{i=1}^{\infty} M^{i-1} T_{22}^{-1} Q_2 
         \Big( 
            \Psi E_t\big\{ \epsilon_{t+i} \big\} + 
            \Pi E_t \big\{\eta_{t+i} \big\}
         \Big) = 0
\]
which imposes a restriction on $ \epsilon_t,\eta_t $. If we go back to \eqref{eq:QZsims_eq01} and take into account $
W_{2,t} = 0$  in the second block, this imposes
%%%
\begin{equation}
   \label{eq:QZsims_epislon_x_eta}
   \underbrace{Q_2 \Psi}_{ n_u \times \ell } \ \underbrace{\epsilon_t}_{ \ell \times 1 } 
   \quad + \quad 
   \underbrace{Q_2 \Pi}_{ n_u \times k }     \ \underbrace{\eta_t}_{k\times 1} = 0
\end{equation}
%%%
As before, existence problems arise if the endogenous shocks $ \eta $ cannot adjust to offset the exogenous shocks
$\epsilon $ in \eqref{eq:QZsims_epislon_x_eta}. Note that the assertion in \eqref{eq:QZsims_epislon_x_eta} is only
possible because of the degree of freedom to choose $\eta$, otherwise it requires that exogenously evolving events
always satisfy a deterministic equation.

% Subsection general_case (end)
%..................................................................................................

% Subsection sims (end)
%..........................................................................................................................

% SECTION LINEAR_RATIONAL_EXPECTATIONAL_DIFFERENCE_EQUATIONS (END)
%..................................................................................................


% =================================================================================================
%----------------------------------------- REFERENCE -------------------------------------------
% =================================================================================================
\clearpage
\newpage
%%%%% MAC
% \bibliographystyle{/Users/felipealves/Dropbox/TexFolder/plainnat2}
% \bibliography{/Users/felipealves/Dropbox/TexFolder/sample2}
%%%%% PC
\bibliographystyle{C:/Users/falves/Dropbox/TexFolder/plainnat2}
\bibliography{C:/Users/falves/Dropbox/TexFolder/2nd_year_paper}
% \bibliography{C:/Users/falves/Dropbox/TexFolder/}

\newpage
\appendix
%%%%%%% Perturbing the value Function  
%------------------------------------------------------------------------------------------------------------
\section{Perturbing the value Function} % (fold)
\label{sub:perturbing_the_value_function}

function in Reiter: \texttt{BellmanIterGivenPolicy1dd}

We adopt the following approximation to perturb the value function
%%%
\begin{equation}
   \label{eq:interpol}
   \text{itp} \Big(  \bsy{V}, v^{*} \Big)(z_j, \tilde{x} ) = v^*( z_j, \bsy{x} ) + 
      \frac{ \tilde{x} - \bar{x}_i }{ \bar{x}_{i+1} - \bar{x}_i }    \Big( \bsy{V}_{j, i+1} - v^{*}( z_j, \bar{x}_{i+1}) \Big) +
      \frac{ \bar{x}_{i+1} -\tilde{x} }{ \bar{x}_{i+1} - \bar{x}_i } \Big( \bsy{V}_{j, i}   - v^{*}(z_j, \bar{x}_i) \Big)
\end{equation}
In steady state, $\bsy{V}_t[ j,i ] = v^* ( z_j,x_i ) $ so that in steady state the whole expression reduces to simply $ v^*( a, \tilde{x} ) $.
% Subsection perturbing_the_value_function (end)
%............................................................................................................

%%%%%%% How to do the perturbation of policy  
%------------------------------------------------------------------------------------------------------------
\section{How to do the perturbation of policy} % (fold)
\label{sub:how_to_do_the_perturbation_of_policy}

\texttt{policyPtyerturb1dd} for perturbation of policies

The policy outside steady state must satisfy
%%%
\begin{equation}
   \label{eq:maximization_prob}
   x_{t}'\Big( z, x \Big) = \argmax_{x'} \Bigg\{ U(z,x,x'; X_t) + \beta \ 
      \mathbb{E}_t
      \bigg[ 
         \sum_{ z' \in \mathcal{Z} } \Pi(z,z') \text{itp} \Big[ \bsy{V}_{t+1} , v^* \Big]( z', x' ) 
      \bigg]
      \Bigg\}
\end{equation}
%%%

Since it is optimal, it must satisfy the \emph{foc} for all realizations of aggregate state
%%%
\begin{equation}
   \label{eq:foc}
   % f(z,x; X_t, X_{t+1}) \defeq 
   \mathbb{E}_t 
   \left[
   U_{x'} \Big( z, x, \bsy{x}_t'(z,x); \bsy{X}_t \Big) + \beta
      \sum_{z \in \mathcal{Z} }  \Pi(z,z')
      \Bigg( 
         \frac{ \partial v^*(z', \bsy{x}_t'(z,x) ) }{ \partial x' } + 
         \frac{ d \bsy{V}_{t+1}[j,i+1 ] - d \bsy{V}_{t+1}[j,i] }{ \bar{x}_{i+1} - \bar{x}_{i} }
       \Bigg)
   \right] = 0
   \tag{dvdc}
\end{equation}
We know that at steady-state the policy computed in Step XX satisfy
%%%
\begin{equation*}
   \label{eq:focstst}
   0 = U_{x'} \Big( z, x, x'(z,x; X^*); X^* \Big) + \beta 
      \sum_{z \in \mathcal{Z} }  \Pi(z,z')
      \Bigg( 
         \frac{ \partial v^* \Big(z', x'(z,x; X^*) \Big) }{ \partial x' } +
         \frac{ d \bsy{V}^*_{j,i+1} - d \bsy{V}^*_{j,i} }{ \bar{x}_{i+1} - \bar{x}_{i} }
       \Bigg)
\end{equation*}
%%%
while the \emph{soc} again at steady state values satisfy
%%%
\begin{equation}
   \label{eq:soc}
   U_{x'x'} \Big( z,x, x'(z,x; X^*) ; X^* \Big) + \beta 
         \sum_{z \in \mathcal{Z} }  \Pi(z,z')
         \frac{ \partial^2 v^*\Big( z', x'(z,x; X^*) \Big)}{ \partial {x'}^2}
   \tag{d2vdc2}
\end{equation}
%%%
Hence, to compute how the policy changes given variations on other equilibrium quantities we can use
the \emph{Implicit function theorem} to get
%%%
\begin{equation}
   \label{eq:}
   \frac{ \partial x' \Big( z,x ; X^* \Big) }{ \partial X }  = - \frac{1}{ \text{d2vdc2} } \Big( \text{d2vdc2.der} \Big)
\end{equation}
%%%   
% Subsection how_to_do_the_perturbation_of_policy (end)
%..................................................................................................

%%%%%%%%%%%%%%%%%%%%%%%%%%%%%%%%%%%%%%%%%%%%%%%%%%%%%%%%%%%%%%%%%%%%%%%%%%%%%%%%%%%%%%%%%%%%%%%%%%%
%%%%%%%%%%%%%%%%%%%%%%%%%%%%%%%%%%%%%%%%%%%%%%%%%%%%%%%%%%%%%%%%%%%%%%%%%%%%%%%%%%%%%%%%%%%%%%%%%%%
%%% End document

\end{document}   
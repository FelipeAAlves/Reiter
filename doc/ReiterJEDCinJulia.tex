
%%%%%%%%%%%%%%%%%%%%%%%%%%%%%%%%%%%%%%%%%
% Draft
% 
%
% NOTE : BUILD with latexmk 
%%%%%%%%%%%%%%%%%%%%%%%%%%%%%%%%%%%%%%%%%
\documentclass[a4paper,10pt]{article}  %scrartcl

\input{C:/Users/falves/Dropbox/TexFolder/preamble.tex} 

%% Aditional packaged
\usepackage[ntheorem]{empheq}
\usepackage{minted} % HOW to compile with the minted option??

\title{User Guide 01 - Krusell-Smith \vspace{-1.00em}}
% {
%         \vspace{-0in
%         \usefont{OT1}{bch}{b}{n}
%         \normalfont \normalsize \ \[5pt]
%         \horrule{0.5pt} \[0.0cm]
%         \huge Referee report on ``Learning, Confidence, and Business Cycles''  \[-0.5cm]
%         \horrule{2pt} \[-0.5cm]
% }
\author{
        \normalfont\large Felipe Alves \\[-2.5pt]       \normalsize
        \today
}
\date{ \vspace{-3em} }

% *************************************************************************************************************
% ************************************************                *********************************************
% *************************************************************************************************************
% NOTE : BUILD with latexmk 

\begin{document}
\maketitle

\begin{abstract}
   These notes presents an step-by-step on Reiter's Projection+Perturbation approach to solve Heterogeneous Agents model with aggregate
   uncertainty. The method is discussed on the context of \citet{krusell_smith} model. The presentation draws heavily on
   \citet{winberry}, although the implementation differs on some relevant aspects.
\end{abstract}

%**************************************************************************************************************************
%%%%%%% INTODUCTION
%--------------------------------------------------------------------------------------------------------------------------
% \section{Intoduction} % (fold)
% \label{sec:intoduction}

% SECTION INTODUCTION (END)
%..........................................................................................................................

%**************************************************************************************************************************
%%%%%%% MODEL
%--------------------------------------------------------------------------------------------------------------------------
\section{Model} % (fold)
\label{sec:model}

\textbf{Households} \quad There is a continuum of households indexed by $ j\in [0,1] $, each with preferences over consumption
$ c_{jt} $ represented by the utility function
\begin{equation*}
   \label{eq:util}
   \mathbb{E}  \sum_{t=0}^{\infty} \beta^t \frac{ c_{jt}^{1- \sigma} -1}{ 1- \sigma }
\end{equation*}
where $ \beta $ is the subjective discount factor and $ \frac{1}{\sigma} $ is the elasticity of intertemporal substitution.
%
Each household supplies inelastically $ \epsilon_{jt} $ efficiency units of labor in the labor market at market price $w_t$.
This idiosyncratic shock is distributed independently across households but for each one particular
household it follows a two-state Markov process with values $\epsilon \in E\defeq  \big\{ 0,1 \big\} $ and transition probabilities $ \Pi $.
Therefore households with idiosyncratic shock $ \epsilon_{j,t} = 0 $ don't supply labor but receive
unemployment benefits in the value of $ bw_t $, while households with $\epsilon_{j,t}=1$ supply their whole unit of efficiency labor
and enjoy after-tax labor earnings of $ w_t ( 1-\tau ) $.

Asset markets are incomplete.
Agents have access only to one type of asset with rate of return $r_t$, potentially stochastic
but independent of the individual state. Moreover, they asset holdings are subject to a borrowing constraint.


Therefore, the flow \textbf{budget constraint} at time $t$ reads
\begin{align*}
   \label{eq:budget_const}
   c_{j,t} + a_{j,t+1} & = \Big( 1 + r_t \Big) a_t + w_t \Big( \epsilon_{j,t} ( 1-\tau ) + ( 1-\epsilon_{j,t} )b \Big) \\
   a_{j,t+1}           & \ge \underline{a}
\end{align*}

\textbf{State Variables} \
Before proceeding let's think what are the relevant states for the household problem and define some notation
while we are at it.
Idiosyncratic levels of asset holdings and employment shock $ ( a, \epsilon ) $ must be in the state, since they enter
on the budget constraint. These are the \emph{idiosyncratic states} of the agent. But note that this is not enough.

The price vector $( r_t, w_t )$ also appears on the budget constraint and, in the recursive competitive equilibrium
definition to be given below, must be themselves a function of the household's state variables.
The assumptions on supply side make them a function of aggregate capital $K_t$ and technological shock $z$ in equilibrium.
So why not have $(K_t,z_t)$ themselves in the state?

It turns out that we need to keep track of more than just aggregate capital to compute household's policies.
Note that when deciding how much to consume/save agents must also predict next period prices distribution - therefore
next period capital - which in this heterogeneous agent model is not determined only by the aggregate level of present capital stock.
Since agents with have different savings rates depending on their idiosyncratic states, predicting next period capital
requires knowledge of the whole distribution of agents over \emph{idiosyncratic states}.
When we are solving only for a stationary equilibrium, the distribution over \emph{idiosyncratic states} is invariant,
making aggregate capital and hence prices constant. But in the presence of aggregate shocks, as is the case here,
aggregate capital and hence the prices will vary which forces household to keep track of the whole distribution
to be able to predict next period capital/prices.
%%
Therefore, the \emph{aggregate states} of household problem are given by $ (z_t, \mu_t ) $ where
$ \mu_t $ denotes the time $t$ measure of households across individual states.


\textbf{Government} \ Government finances unemployment insurance payments by taxing labor earnings with constant tax $\tau$.
Government's budget constraint is balanced each period implying
\begin{equation*}
   \label{eq:gov_budget}
   \tau \int \mu_t( da,1 ) = b \int \mu_t( da,0 )
\end{equation*}
or simply, $ \tau = \frac{b ( 1-L )}{L} $ where $ L \defeq \mu_t( 1, A) $.

\bigskip
\textbf{Firms} \ There is a representative firm which produces output $ Y_t $ according to the production function
\begin{equation*}
   \label{eq:prod_fnc}
   Y_t = e^{z_t} K_t^{\alpha} L_t^{ 1- \alpha }
\end{equation*}
where $ z_t $ is an aggregate productivity shock, $K_t$ is the aggregate capital stock and $ L_t $ is aggregate labor.

In equilibrium, factor prices are determined competitively and given by the marginal products of inputs
\begin{align*}
   \label{eq:factor_prices}
   r_t & = \alpha e^{z_t} K_t^{ \alpha-1 } L_t^{ 1- \alpha } - \delta \\
   w_t & = ( 1- \alpha ) e^{z_t} K_t^{ \alpha } L_t^{ - \alpha }
\end{align*}
%
where $ \delta $ is the depreciation rate. Aggregate TFP follows an AR(1) process
\begin{equation}
   \label{eq:tfp}
   z_{t+1} = \rho z_t + \sigma_z \omega_{t+1}, \qquad \omega_{t+1} \sim \mathcal{N}( 0,1 )
\end{equation}

%%%%%%%%%%%%%%%%%%%%%%%%%%%%%%%%%%%%%%%%%%%%%%%%%%%%%%%%%%%%%%%%%%%%%%%%%%%%%%%%%%%
\textbf{Equilibrium} \\
A \textbf{Recursive Competitive Equilibrium} is a list of functions 
\[
   \Big\{
   c\big( a,\epsilon ; z, \mu \big) ,a'\big( a,\epsilon ; z, \mu \big),
   r\big( z,\mu \big), w\big( z,\mu \big),
   \Psi (\mu,z)
   \Big\} 
\]
\begin{enumerate}
   %%
   \item ( Household Optimization ) \\
         Taking price functions $ r(\cdot), w( \cdot ) $ and the law of motion $\Psi(\cdot)$ as given,
         $ \Big\{ c( a,\epsilon; z, \mu ) , a'( a,\epsilon ; z, \mu ) \Big\}$ must satisfy
         \begin{align}
              & c( a, \epsilon ; z,\mu )^{ - \sigma } \ge
            \beta \mathbb{E} \bigg\{
            \Big( 1+ \tilde{r} \Big( z', \Psi( z, \mu ) \Big) \Big) c\Big( a'( a, \epsilon; z, \mu ),
            \epsilon'; z', \Psi( z,\mu ) \Big)^{- \sigma}
            \bigg\} \label{eq:euler} \\
              & c( a, \epsilon ; z,\mu ) + a'( a,\epsilon; z, \mu ) = \big( 1 + \tilde{r}(z, \mu ) \big)
            a + w \big( z, \mu \big) \tau( \epsilon )  \label{eq:budget_constraint}
         \end{align}
         with the euler holding with equality if $ a'\big( a, \epsilon; z , \mu \big) > \underline{a} $.

         %%
   \item ( Firm Optimization ) \\
         Given price function $ r( \cdot ) , w ( \cdot ) $, firms choose capital $K$ and labor $L$ according to
         \begin{align}
            r( z, \mu ) & = \alpha e^{z} K^{ \alpha-1 } L^{ 1- \alpha } - \delta \\
            w( z, \mu ) & = ( 1- \alpha ) e^{z} K^{ \alpha } L^{ - \alpha }
         \end{align}

         %%
   \item ( Market Clearing ) \\~\\
         Labor market clears:
         \begin{equation}
            \label{eq:labor_mkt}
            L = \int \epsilon \mu( da, d\epsilon )
         \end{equation}
         %%%%%
         Asset market clears:
         \begin{equation}
            \label{eq:asset_mkt}
            K = \int a \mu( da, d \epsilon )
         \end{equation}
         %%%%%
         Good market clear:
         \begin{equation}
            \label{eq:asset_mkt}
            \int c( a, \epsilon ; z, \mu) \mu( da, d \epsilon ) + \int a'( a,\epsilon ; z, \mu) \mu( da, d \epsilon )
            = \exp{( z )} F( K,L ) + ( 1- \delta ) K
         \end{equation}
         %%
   \item Government budget constraint is satisfied

         %%
   \item ( Evolution of distribution ) \\
         For any $ \epsilon' \in E $ and measurable set $ \mathcal{A} \subseteq A $, next period distribution implied by the
         law of motion $ \Psi $ is consistent with the households' policies
         \begin{equation}
            \label{eq:law_of_motion}
            \Psi( z, \mu ) \Big( \mathcal{A} \times \{\epsilon'\} \Big) = \int
            %
            \ \mathbf{1} \Big\{  a'( a, \epsilon; z, \mu ) \in \mathcal{A} \Big\} \ \pi( \epsilon' | \epsilon ) \ \mu( da, d \epsilon )
            %
         \end{equation}
\end{enumerate}

% SECTION MODEL (END)
%..........................................................................................................................

\newpage
%**************************************************************************************************************************
%%%%%%% METHOD
%--------------------------------------------------------------------------------------------------------------------------
\section{Method} % (fold)
\label{sec:method}

At a broad level, the solution method includes three steps
\begin{enumerate}
   %%
   \item Approximate the infinite dimensional equilibrium objects, which in our example involves the
         policy functions of the household and cross-sectional distribution of individual states, by some
         finite dimensional object. This allows us to have a finite parametrization of the model that we
         shall refer to as the \emph{discrete model}.

         %%
   \item Compute the stationary equilibrium of the \emph{discrete} model. This will take into account
         the idiosyncratic uncertainty but \textbf{not} the uncertainty coming from aggregate shocks.

         %%
   \item Linearize the discrete model equations with respect to our finite parametrization around
         steady-state and compute the rational expectation solution of the linearized system taking into
         account aggregate shocks.
\end{enumerate}

%%%%%%% Finite-Dimensional Approximation
%--------------------------------------------------------------------------------------------------------------------------
\subsection{Finite-Dimensional Approximation} % (fold)
\label{sub:finite_dimensional_approximation}
Looking at the equilibrium definition we see that there are two infinite-dimensional objects that need
to be approximated if we intend to solve the model using \citet{reiter} method.

First, we need a finite representation of household's individual state distribution. I will offer two
options here, one that keeps track of histogram defined on finite number of points of the state space and
another that tries to approximate the distribution by a density function of a finite dimensional family.
Second, household's decision rules characterized by \eqref{eq:euler} are the solution of an functional equation and
must also be approximated.

%..........................................................................................................................
\textbf{Distribution I} \
The first option comes from \citet{reiter} and involves approximating the distribution of households by a
finite number of mass points on a predefined grid for $ \mathcal{A} \defeq \big\{ a_j \big\}_{j=1}^{N_a} $ and productivity $E$.
%%
Let $ \Phi( a_j, \epsilon ) $ denote the fraction of households with productivity level $ \epsilon $ and
asset holdings $ a_j $. The evolution of this distribution is implied by the decision rules as follows.
For any pair $ ( a_{j'}, \epsilon' ) \in \mathcal{A}\times E $, next period mass at the point is given by
%%
\begin{equation}
   \Phi' ( a_{j'}, \epsilon' ) = \sum_{ \epsilon \in E } \pi( \epsilon, \epsilon' )
   \sum_{ j=1 }^{N_a}
   \omega_{j,\epsilon, j'}
   \times  \Phi(a_j, \epsilon) \\
\end{equation}
where $\omega_{j,\epsilon, j'}$ is a function of savings of the household
\begin{equation*}
   \label{eq:weight}
   \omega_{j,\epsilon, j'} = \begin{cases}
   \frac{ a' - a_{j'-1} }{ a_{j'} - a_{j'-1} } & \text{ if }
   a'( a_j, \epsilon ; z, \mu ) \in \Big[ a_{j'-1}, a_{j'} \Big] \\ \ & \ \\
   \frac{ a_{j'+1} - a' }{ a_{j'+1} - a_{j'} } & \text{ if }
   a'( a_j, \epsilon ; z , \mu) \in \Big[ a_{j'}, a_{j'+1} \Big] \\ \ & \ \\
   0 & \text{o.w.}
   \end{cases}
\end{equation*}

\todo[inline]{Here's something I might need to add.}
% Note that I have already replaced the true aggregate state $ ( z,\mu ) $ with the approximate aggregate state
% $ ( z, \Phi ) $ in the decision rule of the household.

\textbf{Distribution II} \
The second alternative follows \citet{winberry} and tries to approximate the distribution over assets
conditional on a realization of labor productivity $ \epsilon $ - $\mu( da,\epsilon )  $ -
using the following parametric family
%%%
\begin{equation}
   \label{eq:density}
   g_{\epsilon} \Big( a ; g_{\epsilon}^0, \big\{ g^i_{\epsilon} ,m^i_{\epsilon}\big\}_{i=1}^{n_g} \Big)
   = g_{\epsilon}^{0} \exp \left\{
   g_{\epsilon}^1( a - m^1_{\epsilon} ) +
   \sum_{i=1}^{n_g} g^1_{\epsilon} \bigg[ ( a - m^1_{\epsilon} )^i - m^i_{\epsilon}\bigg]
   \right\}
\end{equation}
%%%
where $ n_g $ denotes the order of approximation, $ \big\{ g^i_{\epsilon} \big\}_{i=0}^{n_g}$ are the parameters of the distribution
which must be consistent with the centralized moments $ \mathbf{m}_{\epsilon} \defeq \big\{ m^i_{\epsilon}\big\}_{i=1}^{n_g} $, that is,
%%%
\begin{equation}
   \label{eq:consitency}
   %%%
   \begin{split}
      m^1_{\epsilon} & = \int a g_{\epsilon}( a ) da                                  \\
      m^i_{\epsilon} & = \int \big( a-m^1_{\epsilon} \big)^i g_{\epsilon}( a )da 
   \end{split}
\end{equation}
%%%

Turns out that finding a solution of this non-linear system of equation is hard and without
an good initial guess convergence is uncertain. To solve this issue we follow the suggestion by \citet{algan} and find the coefficients - except for $ g^0_{\epsilon} $ - as the solution to the following minimization problem
\begin{equation*}
   \label{eq:min_denscoeff}
   \min_{ \quad \big\{ \rho_{ \epsilon}^i\big\}_{i=1}^{n_g} } \quad \int g \Big( a; 1, \big\{\rho_{ \epsilon }^i ,m^i_{\epsilon}\big\}_{i=1}^{n_g} \Big) da
\end{equation*}
To see that, note that the first order conditions of the problem correspond exactly to consistency conditions
\eqref{eq:consitency}. Turns out that solving this minimization problem is way easier than solving the system
\eqref{eq:consitency} and it works even without a good initial condition.

This approximation reduces the infinite-dimensional distribution $\mu$ to a set of moments
$ \big\{\mathbf{m}_{\epsilon}\big\}_{\epsilon \in E} $. But note that we still need to derive the law of motion
in terms of this approximate aggregate state. Although current density together with decision rules pin down the
distribution for next period, the distribution has no reason to be an element of the parametric family
\eqref{eq:density}.
That in principle presents us with a problem. But note that our reduced state only requires us to keep track of
next period moments, from which we can derive next period approximated density through consistency conditions.
Next period moments in turn are determined as
%
\begin{align*}
   {m_{\epsilon}^1}' & = \sum_{ \tilde{\epsilon} \in E }
   \frac{ \pi ( \tilde{\epsilon} ) \pi( \tilde{\epsilon}, \epsilon ) }{ \pi( \epsilon ) }
   \int a'\Big( a, \tilde{\epsilon}; z, \mathbf{m} \Big) g_{ \tilde{\epsilon} }(a) da \\
   %%%
   {m_{\epsilon}^i}' & =   \sum_{ \tilde{\epsilon} \in E }
   \frac{ \pi ( \tilde{\epsilon} ) \pi( \tilde{\epsilon}, \epsilon ) }{ \pi( \epsilon ) }
   \int \Big( a'\Big( a, \tilde{\epsilon}; z, \mathbf{m} \Big)  - {m_{\epsilon}^1}' \Big)^i g_{ \tilde{\epsilon} }(a) da
\end{align*}
%
In practice, I compute the integrals using a Gauss-Legendre quadrature, which specifies nodes
$\big\{ a_j \big\}_{j=1}^{n_q}$ and weights $\big\{ \omega_j \big\}_{j=1}^{n_q}$ and replaces the integrals
with finite sums
\begin{equation*}
   \label{eq:aaa}
   \int a'( a, \tilde{\epsilon}; z, \mathbf{m} ) g_{ \tilde{\epsilon} }(a) da \approx
   \sum_{j=1}^{n_q} \omega_j a'( a_j, \tilde{\epsilon}; z, \mathbf{m} ) g_{ \tilde{\epsilon} }(a_j)
\end{equation*}
%..........................................................................................................................

\textbf{Household Decision Rules}

Optimality conditions in terms of the new (finite-dimensional) aggregate state are given by
\begin{align*}
     & c \Big( a, \epsilon; z, \Phi \Big)^{ - \sigma } \ge \beta \mathbb{E}_{z'|z}
   \Bigg\{
   \big( 1+\tilde{r}( z', \Phi')  \big)
   \sum_{ \epsilon' \in E } \pi( \epsilon'| \epsilon )  c \Big( a'( a, \epsilon; z, \Phi ), \epsilon'; z', \Phi' \Big)^{- \sigma}
   \Bigg\} \\
     & c\big( a, \epsilon; z, \Phi \big) + a'\big( a, \epsilon; z, \Phi \big) =
   \big( 1 + r( z, \Phi ) \big) a + w ( z,\Phi ) \tau( \epsilon )
\end{align*}
where conditional expectation of future marginal utility has been broken into two pieces: (i) the expectation
with respect to idiosyncratic shocks is taken explicitly, (ii) expectation with respect to aggregate shocks is only
implicitly in the expectation operator and will be taken into account only during the perturbation step.

Household's savings behavior is characterized by a critical level of assets $ \chi_{ \epsilon } $
at which the borrowing constraint starts binding and a smooth function for $ a > \chi_{ \epsilon } $.
I follow \citet{reiter} and approximate the savings rule by a piecewise linear spline with knot points
$ a_{ i, \epsilon}  = \chi_{ \epsilon } + x_i , \ i = 1, \ldots, n_s$ with
$ 0 = x_1 < \ldots < x_{n_s} $.

For a given aggregate state $ ( z, \Phi ) $, household's savings policy is then represented by
$n_{\epsilon} \times n_s$ coefficients giving for each idiosyncratic shock $ \epsilon $ the critical
level $ \chi_{\epsilon} $ and savings at $ a_{2, \epsilon}, \ldots, a_{n_s, \epsilon} $
- note that savings at $ a_{1,\epsilon} $ equals $\underline{a}$ by construction.

These coefficients are collected into the vector $ \boldsymbol{\theta}( z, \Phi ) $ and the approximated
consumption function is then written as $ \hat{c} \big( a, \epsilon ;\  \bsy{\theta}(z, \Phi) \ \big) $.
%%%
Given this approximation choice, household's optimality conditions are approximated using
collocation, which forces \eqref{eq:euler}-\eqref{eq:budget_constraint} to hold exactly on the set of nodes
$ \big\{ a_{i, \epsilon }, \epsilon \big\}_{i\in\{1,\ldots, n_s\},\epsilon\in E}$
\begin{align*}
     & \hat{c} \Big( a_{i,\epsilon}, \epsilon; \ \bsy{\theta}( z,\Phi ) \ \Big)^{ - \sigma } = \beta \mathbb{E}_{z'|z}
   \Bigg\{
   \big( 1 + \tilde{r}( z', \Phi')  \big)
   \sum_{ \epsilon' \in E } \ \pi( \epsilon'| \epsilon ) \ \hat{c} \Big( \hat{a}'( a_{i,\epsilon}, \epsilon; \ \bsy{\theta}( z,\Phi ) \ ),
   \epsilon' ; \ \bsy{\theta}( z', \Phi' ) \ \Big)^{- \sigma}
   \Bigg\} \\
     & \hat{c} \Big( a_{i,\epsilon}, \epsilon; \ \bsy{\theta}( z,\Phi ) \ \Big) +
   \hat{a}'\Big( a_{i,\epsilon}, \epsilon; \ \bsy{\theta}( z,\Phi ) \ \Big) =
   \big( 1 + \tilde{r}( z, \Phi ) \big) a_{i,\epsilon} + w( z, \Phi ) \tau( \epsilon )
\end{align*}

The notation is intended to make clear how differently we treat the dependence of policy on
idiosyncratic and aggregate states. Policy dependence on individual states is explicitly consider and
parametrized by finite-dimensional vector $  \bsy{\theta}(z, \Phi)  $, whose value at steady state
$\bsy{\theta}^*$ is determined by our collocation scheme. Policy depends on aggregates states only
insofar as that parametrization varies with $ (z, \Phi) $, and that dependence is solved
at the perturbation step only.

\begin{quote}
   Mkcay and Reis: We approximate the policy rules for savings by piece-wise linear splines with
   100 knot point each. We deal with the borrowing constraint in the approximation of the policy
   functions by parameterizing the point at which the borrowing constraint starts binding
\end{quote}

\textbf{Approximate Equilibrium Conditions} \\ Given all these approximations, our definition of the
\textbf{recursive competitive equilibrium} is reduced to a finite set of equations.
To collect these conditions using a compact notation define the state vector $ \mathbf{x} = (\Phi,z, K ) $ which contains
$2 + \big(N_a \times n_{\epsilon} \big)$
predetermined variables
and the control vector $ \mathbf{y} = ( \bsy{\theta}, r, w )  $ of size
$2 +\big(n_s \times n_{\epsilon}\big)$.
Therefore, the our discrete equilibrium can be resumed by a system of nonlinear equations
\begin{equation}
   \label{eq:SYSTEM}
   \mathbb{E}_{z'|z} \Big[ f(\mathbf{y}',\mathbf{y},\mathbf{x}',\mathbf{x}, \omega') \Big] = 0
\end{equation}

where $f$ is given by in the case we choose to represent the distribution by histogram (\textbf{Distribution I})
\begin{empheq}[left=\empheqlbrace]{gather*}
   %% transition
   \Phi' ( a_{j'}, \epsilon' ) - \sum_{ \epsilon \in E } \pi( \epsilon, \epsilon' )
   \sum_{ a_j \in \mathcal{A} }
   \Big( \omega_{j,\epsilon, j'} \Phi(a_j, \epsilon) \Big) \quad a_{j'}\in \mathcal{A}, \epsilon \in E
   \\ \ \\
   %% Exogenous
   z'  - \rho_z z - \sigma_{z} \omega_z' \\ \ \\
   % ( \tau^k )'  - \rho_k \tau^k - \sigma_{k} \omega_k' \\ \ \\
   %
   %% Consistency
   K' \ - \sum_{\epsilon \in E}\sum_{ j=1 }^{N_a} a_j \Phi'( a_j, \epsilon ) \\ \ \\
   %
   %% Government Budget Constraint
   % T \bar{L} - \tau^k  K \\ \ \\
   %
   %% Prices
   r - \alpha e^{z} K^{ \alpha-1 } L^{ 1- \alpha } + \delta \\
   w - ( 1- \alpha ) e^{z} K^{ \alpha } L^{ - \alpha }   \\ \ \\
   %
   %% Euler eq
   \hat{c} \Big( a_{i,\epsilon}, \epsilon; \ \bsy{\theta} \ \Big)^{ - \sigma } = \beta
   \Bigg\{
   \big( 1 + r'\big)
   \sum_{ \epsilon' \in E } \ \pi( \epsilon'| \epsilon ) \ \hat{c} \Big( \hat{a}'( a_{i,\epsilon}, \epsilon; \bsy{\theta}  ),
   \epsilon' ; \ \bsy{\theta}' \ \Big)^{- \sigma}
   \Bigg\} \\
   %
   \qquad \hat{c} ( a_{i,\epsilon}, \epsilon; \ \bsy{\theta} \ ) + \hat{a}'( a_{i,\epsilon}, \epsilon; \bsy{\theta}  ) =
   ( 1+r ) a_{i, \epsilon} + w \tau ( \epsilon ) ,
   \qquad i =1, \ldots, n_s \text{ and } \epsilon \in E \\
\end{empheq}
{\color{RubineRed} \textsc{Important}:} It is crucial to have the consistency condition in terms of next period assets.
Problems with stable/unstable roots otherwise. 
% subsection finite_dimensional_approximation (end)
%..........................................................................................................................

%%%%%%% Stationary Equilibrium
%--------------------------------------------------------------------------------------------------------------------------
\subsection{Stationary Equilibrium} % (fold)cd
\label{sub:stationary_equilibrium}

In terms of the notation presented above, a \emph{stationary equilibrium} of the \emph{discrete} model
are values $\mathbf{x}^*, \mathbf{y}^*$ such that $f(\mathbf{y}^*,\mathbf{y}^*,\mathbf{x}^*,\mathbf{x}^*, 0)=0$.
Note that this is just a nonlinear system with as many equations ans unknowns - $6 + \big( (N_a+n_s) \times n_{\epsilon}\big)$.
However, the system is too large to be solved with usual numerical algorithms and we need a more stable scheme to search for the equilibrium.

The following algorithm seems to work in practice. Set up an initial guess for capital $K^0$
\begin{enumerate}

   %%%
   \item Back out market-clearing factor prices $r^0 = \alpha K^{ \alpha -1 } L^{1-\alpha} - \delta $ and
         $ w^0 = (1-\alpha) K^{\alpha} L^{-\alpha}$

         %%%
   \item Given fixed prices $ \big( r^0, w^0 \big) $, we can solve the dynamic problem of the agent for
         $\bsy{\theta}^*_0$

         %%%
   \item Using implied decisions, solve for \emph{invariant distribution}, i.e.
         \begin{equation*}
            \Phi^0 ( a_{j'}, \epsilon' ) - \sum_{ \epsilon \in E } \pi( \epsilon, \epsilon' )
            \sum_{ a_j \in \mathcal{A} }
            \Big( \omega_{j,\epsilon, j'} \big( \bsy{\theta}_0^{*} \big) \times \Phi^0(a_j, \epsilon) \Big)
         \end{equation*}

         %%%
   \item Compute the aggregate supply of capital deduced the invariant distribution
         \[
            A^0 = \sum_{\epsilon \in E} \ \sum_{ j=1 }^{N_a} \ a_j \ \Phi^0 ( a_j, \epsilon )
         \]

         %%%
   \item Check $ \lvert A^0 - K^0 \rvert < \epsilon $. If not update the guess for capital $ K^1 $.%
         \footnote{The updating rule may come from a bisection or newton algorithm.}
\end{enumerate}
% subsection stationary_equilibrium (end)
%..........................................................................................................................


%%%%%%% Linearization
%--------------------------------------------------------------------------------------------------------------------------
\subsection{Linearization using Automatic Differentiation} % (fold)
\label{sub:linearization_using_automatic_differentiation}

As is stated on \emph{wiki}

\begin{quote}
   \em
   Automatic differentiation (AD) is a set of techniques to numerically evaluate the derivative of a function specified by a computer program. AD exploits the fact that every computer program, no matter how complicated, executes a sequence of elementary arithmetic operations (addition, subtraction, multiplication, division, etc.) and elementary functions (exp, log, sin, cos, etc.). By applying the \textbf{chain rule} repeatedly to these operations, derivatives of arbitrary order can be computed automatically, accurately to working precision, and using at most a small constant factor more arithmetic operations than the original program.
\end{quote}

The key idea of AD in to use basic derivative rules from calculus, such as the chain rule, in a numerical environment.
The derivatives are then computed together with the evaluation steps and combined with other derivatives using these rules.

As an example, consider the function $f$ defined by $x \rightarrow x \sin x^2$. When evaluating this function at
$x=3$ you computer will compute the sequence of values on the left of table

\begin{table}[H]
   \centering
   \begin{tabu} to 0.8\textwidth { | X[c] | X[c] | } \hline
      $x = 3$           &  $\dot{x} = 1$                       \\ \hline
      $w_1 = x^2$       &  $\dot{w}_1 = 2 x \dot{x}$           \\ \hline
      $w_2 = \sin(w_1)$ &  $\dot{w}_2 = \cos (w_1) \dot{w}_1$  \\ \hline
      $w_3 = x w_2$     &  $\dot{x}w_2 + x \dot{w}_2$          \\ \hline
   \end{tabu}
\end{table}

But for each computation in the left, we can compute the derivatives that aggregate up to the derivative of
the function at last line. What AD does is make derivative computations to happen automatically
when evaluating or interpreting the values on the left side.

These ideas are implemented in \textsc{Matlab} using object-oriented programming (OOP) features to define a new class of value-and-derivative objects. Methods are then associated
to such objects in order to implement standard calculus derivative rules with built-in chain rule.
In particular, methods overload the definitions of standard operations and functions, such as $*$ and $\sin$,
to compute and return not only the value of the expression but also its derivative.

To see a simple example, to compute the value and derivative of our function $f$ at 3, we create a
value-and-derivative object \texttt{x} with attributes $v =3$ (value) and $d=1$ (derivative). When we evaluate
the expression \texttt{x * sin( x\^{}2 )} both the left and right hand side of the table are executed and the return value
is also a value-and-derivative object with attributes $v=1.2363$ and $d = −15.9883$.

\textbf{Julia Implementation} \\
Using AD is made easy on \emph{Julia} environment due to the ForwardDiff package which offers an implementation
of forward mode automatic differentiation (AD) in Julia. The package is built upon the \texttt{Dual} type

\begin{minted}{julia}
   type DualNumber{N, T<:Real} <: Real
   value::T
   partials::Partials{N,t}
   end
\end{minted}
where \texttt{Partials} is defined
\begin{minted}{julia}
   type Partials{N,T}
   values::NTuple{N,T}
   end
\end{minted}

The idea here is to enhance any real number $x$ by an additional term $ x' \epsilon $, where $ x'\in \mathbb{R} $ and $ \epsilon $ is an
\emph{infinitesimal} with the property that $ \epsilon^2 =0 $.
All arithmetic operators must then be extended for the augmented algebra.
This new arithmetic will consist of ordered pairs, elements written $ ( x,x') $,
with ordinary arithmetics on the first component, and first order differentiation arithmetic on
the second component.

Defining a new arithmetic to our \texttt{Dual} types is natural in Julia because of its \emph{multiple dispatch} approach.
We just need to add a new method to deal with \texttt{Dual} numbers on all elementary numerical functions
that evaluates both the original function and the derivative of the function. For example
Base.sin should be handled as
\begin{minted}{julia}
   Base.sin(d::Dual) = Dual(sin(value(d)), cos(value(d)) * partials(d))
\end{minted}

ForwardDiff defines the \texttt{Dual} type and does all the overloading so that we don't need to
worry about it when computing derivatives.
Since our function $ f$, no matter how complicated, ends up being composed entirely of these elementary functions, then
the chain rule enables our derivatives to compose as well. Thus we can differentiate $f$
by passing in a Dual number and looking at the output.

ADD simple example on the jupyter \ldots
% subsection linearization_using_automatic_differentiation (end)
%..........................................................................................................................

%%%%%%% Solving Linearized System
%--------------------------------------------------------------------------------------------------------------------------
\subsection{Solving Linearized System} % (fold)
\label{sub:solving_linearized_system}


\paragraph{Klein method.}
Linearizing \eqref{eq:SYSTEM} around the steady-state yields a first-order linear expectational difference equation
system of the form
%%%
\begin{equation}
   \label{eq:linearized_equil_conditions}
   f_{ \mathbf{y}' } E_t \Big[ \underbrace{ ( \mathbf{y}_{t+1} -  \bar{\mathbf{y}}) }_{ \tilde{\mathbf{y}}_{t+1} } \Big] 
   + f_{ \mathbf{y} }   \tilde{\mathbf{y}}_t  
   + f_{ \mathbf{x}' }  E_t \Big[  \tilde{\mathbf{x}}_{t+1} \Big]
   + f_{ \mathbf{x} }   \tilde{\mathbf{x}}_{t}
   = 0
\end{equation}
%%%
which can be set on Klein's form letting $ \Gamma_0 \defeq \big[f_{ \mathbf{x}' } \quad f_{ \mathbf{y}' } \big] $  and
$ \Gamma_1 \defeq - \big[f_{ \mathbf{x} } \quad  f_{ \mathbf{y} } \big] $ so that
%%%
\begin{equation}
   \label{eq:klein}
   \Gamma_0
   E_t \left\{ 
         \begin{bmatrix}
            \mathbf{x}_{t+1} \\ \mathbf{y}_{t+1}
         \end{bmatrix}
      \right\} = 
   \Gamma_1
   \begin{bmatrix}
      \mathbf{x}_{t} \\ \mathbf{y}_t 
   \end{bmatrix}
\end{equation}
%%%

\paragraph{Sims method.}
Let $X_t \defeq ( \tilde{\mathbf{x}}; \ \tilde{\mathbf{y}} )$. To put the model into Sims' format, we need to get rid of
the expectational terms in
\begin{equation*}
   \Gamma_0 \mathbb{E}_t \Big[ X_{t+1} \Big] = \Gamma_1 X_t
\end{equation*}
where $ \Gamma_0, \Gamma_1$ are defined as before. To do so we substitute $ E_{t} \big\{X_{t+1}\big\} $ by the
combination of its \emph{ex post} realizations plus the appropriate forecast errors.\footnote{%
   For each $y_{i,t+1}$, whenever we have $ E_{t} \big\{ y_{i,t+1} \big\}  $  we substitute to $ y_{i,t+1} - \eta_{i,t+1}$, 
   where $\eta_{i,t+1} $  are \textbf{endogenous} expectational errors determined as part of the solution.
   Endogenous states are all pre-determined, so $ E_t\{x_{1,t+1}\} = x_{t+1} $, while exogenous states have exogenous
   forecast error $ \omega_{t+1} $, so we can write $ E_t \{\mathbf{x}_{2,t+1} \} = \mathbf{x}_{2,t+1} - \omega_{t+1}$
   }
Once we do that, we get to Sims' canonical form
\begin{equation}
   \label{eq:gensys}
   \Gamma_0 X_{t+1} = \Gamma_1 X_{t} +  \Psi \omega_{t+1} + \Pi \eta_{t+1}
\end{equation}
which is on Sims' canonical form and can be solved using Sim's \texttt{gensys} algorithm\footnote{
I actually use a modified version of the gensys algorithm that treats the existence and uniqueness more carefully. The code accompanies the
\texttt{Reiter.jl}.\}.

% subsection solving_linearized_system (end)
%..........................................................................................................................

% SECTION METHOD (END)
%..........................................................................................................................


% ============================================================================================================
%------------------------------------------------ REFERENCE ------------------------------------------------
% ============================================================================================================
\clearpage
\newpage
%%%%% MAC
% \bibliographystyle{/Users/Felipe/Dropbox/TexFolder/plainnat2}
% \bibliography{/Users/Felipe/Dropbox/TexFolder/2nd_year_paper}
%%%%% PC
\bibliographystyle{C:/Users/falves/Dropbox/TexFolder/plainnat2}
\bibliography{C:/Users/falves/Dropbox/TexFolder/reiter_project}

\end{document}
%%%%%%%%%%%%%%%%%%%%%%%%%%%%%%%%%%%%%%%%%%%%%%%%%%%%%%%%%%%%%%%%%%%%%%%%%%%%%%%%%%%%%%%%%%%%%%%%%%%
%%%%%%%%%%%%%%%%%%%%%%%%%%%%%%%%%%%%%%%%%%%%%%%%%%%%%%%%%%%%%%%%%%%%%%%%%%%%%%%%%%%%%%%%%%%%%%%%%%%
%%% End document

%%%%%%%%%%%%%%%%%%%%%%%%%%%%%%%%%%%%%%%%%
% Draft
% 
%
% NOTE : BUILD with latexmk 
%%%%%%%%%%%%%%%%%%%%%%%%%%%%%%%%%%%%%%%%%
\documentclass[a4paper,10pt]{article}  %scrartcl

\input{C:/Users/falves/Dropbox/TexFolder/preamble.tex} 

%% Aditional packaged
\usepackage[ntheorem]{empheq}
\usepackage{minted} % HOW to compile with the minted option??

\title{User Guide 01 - Krusell-Smith \vspace{-1.00em}}
% {
%         \vspace{-0in
%         \usefont{OT1}{bch}{b}{n}
%         \normalfont \normalsize \ \[5pt]
%         \horrule{0.5pt} \[0.0cm]
%         \huge Referee report on ``Learning, Confidence, and Business Cycles''  \[-0.5cm]
%         \horrule{2pt} \[-0.5cm]
% }
\author{
        \normalfont\large Felipe Alves \\[-2.5pt]       \normalsize
        \today
}
\date{ \vspace{-3em} }

% *************************************************************************************************************
% ************************************************                *********************************************
% *************************************************************************************************************
% NOTE : BUILD with latexmk 

\begin{document}
\maketitle

\begin{abstract}
   These notes presents an step-by-step on Reiter's Projection+Perturbation approach to solve Heterogeneous Agents model with aggregate
   uncertainty. The method is discussed on the context of \citet{krusell_smith} model. The presentation draws heavily on
   \citet{winberry}, although the implementation differs on some relevant aspects.
\end{abstract}

%**************************************************************************************************************************
%%%%%%% INTODUCTION
%--------------------------------------------------------------------------------------------------------------------------
% \section{Intoduction} % (fold)
% \label{sec:intoduction}

% SECTION INTODUCTION (END)
%..........................................................................................................................

%**************************************************************************************************************************
%%%%%%% MODEL
%--------------------------------------------------------------------------------------------------------------------------
\section{Model} % (fold)
\label{sec:model}

\textbf{Households} \quad There is a continuum of households indexed by $ j\in [0,1] $, each with preferences over consumption
$ c_{jt} $ represented by the utility function
\begin{equation*}
   \label{eq:util}
   \mathbb{E}  \sum_{t=0}^{\infty} \beta^t \frac{ c_{jt}^{1- \sigma} -1}{ 1- \sigma }
\end{equation*}
where $ \beta $ is the subjective discount factor and $ \frac{1}{\sigma} $ is the elasticity of intertemporal substitution.
%
Each household supplies inelastically $ \epsilon_{jt} $ efficiency units of labor in the labor market at market price $w_t$.
This idiosyncratic shock is distributed independently across households but for each one particular
household it follows a two-state Markov process with values $\epsilon \in E\defeq  \big\{ 0,1 \big\} $ and transition probabilities $ \Pi $.
Therefore households with idiosyncratic shock $ \epsilon_{j,t} = 0 $ don't supply labor but receive
unemployment benefits in the value of $ bw_t $, while households with $\epsilon_{j,t}=1$ supply their whole unit of efficiency labor
and enjoy after-tax labor earnings of $ w_t ( 1-\tau ) $.

Asset markets are incomplete.
Agents have access only to one type of asset with rate of return $r_t$, potentially stochastic
but independent of the individual state. Moreover, they asset holdings are subject to a borrowing constraint.


Therefore, the flow \textbf{budget constraint} at time $t$ reads
\begin{align*}
   \label{eq:budget_const}
   c_{j,t} + a_{j,t+1} & = \Big( 1 + r_t \Big) a_t + w_t \Big( \epsilon_{j,t} ( 1-\tau ) + ( 1-\epsilon_{j,t} )b \Big) \\
   a_{j,t+1}           & \ge \underline{a}
\end{align*}

\textbf{State Variables} \
Before proceeding let's think what are the relevant states for the household problem and define some notation
while we are at it.
Idiosyncratic levels of asset holdings and employment shock $ ( a, \epsilon ) $ must be in the state, since they enter
on the budget constraint. These are the \emph{idiosyncratic states} of the agent. But note that this is not enough.

The price vector $( r_t, w_t )$ also appears on the budget constraint and, in the recursive competitive equilibrium
definition to be given below, must be themselves a function of the household's state variables.
The assumptions on supply side make them a function of aggregate capital $K_t$ and technological shock $z$ in equilibrium.
So why not have $(K_t,z_t)$ themselves in the state?

It turns out that we need to keep track of more than just aggregate capital to compute household's policies.
Note that when deciding how much to consume/save agents must also predict next period prices distribution - therefore
next period capital - which in this heterogeneous agent model is not determined only by the aggregate level of present capital stock.
Since agents with have different savings rates depending on their idiosyncratic states, predicting next period capital
requires knowledge of the whole distribution of agents over \emph{idiosyncratic states}.
When we are solving only for a stationary equilibrium, the distribution over \emph{idiosyncratic states} is invariant,
making aggregate capital and hence prices constant. But in the presence of aggregate shocks, as is the case here,
aggregate capital and hence the prices will vary which forces household to keep track of the whole distribution
to be able to predict next period capital/prices.
%%
Therefore, the \emph{aggregate states} of household problem are given by $ (z_t, \mu_t ) $ where
$ \mu_t $ denotes the time $t$ measure of households across individual states.


\textbf{Government} \ Government finances unemployment insurance payments by taxing labor earnings with constant tax $\tau$.
Government's budget constraint is balanced each period implying
\begin{equation*}
   \label{eq:gov_budget}
   \tau \int \mu_t( da,1 ) = b \int \mu_t( da,0 )
\end{equation*}
or simply, $ \tau = \frac{b ( 1-L )}{L} $ where $ L \defeq \mu_t( 1, A) $.

\bigskip
\textbf{Firms} \ There is a representative firm which produces output $ Y_t $ according to the production function
\begin{equation*}
   \label{eq:prod_fnc}
   Y_t = e^{z_t} K_t^{\alpha} L_t^{ 1- \alpha }
\end{equation*}
where $ z_t $ is an aggregate productivity shock, $K_t$ is the aggregate capital stock and $ L_t $ is aggregate labor.

In equilibrium, factor prices are determined competitively and given by the marginal products of inputs
\begin{align*}
   \label{eq:factor_prices}
   r_t & = \alpha e^{z_t} K_t^{ \alpha-1 } L_t^{ 1- \alpha } - \delta \\
   w_t & = ( 1- \alpha ) e^{z_t} K_t^{ \alpha } L_t^{ - \alpha }
\end{align*}
%
where $ \delta $ is the depreciation rate. Aggregate TFP follows an AR(1) process
\begin{equation}
   \label{eq:tfp}
   z_{t+1} = \rho z_t + \sigma_z \omega_{t+1}, \qquad \omega_{t+1} \sim \mathcal{N}( 0,1 )
\end{equation}

%%%%%%%%%%%%%%%%%%%%%%%%%%%%%%%%%%%%%%%%%%%%%%%%%%%%%%%%%%%%%%%%%%%%%%%%%%%%%%%%%%%
\textbf{Equilibrium} \\
A \textbf{Recursive Competitive Equilibrium} is a list of functions 
\[
   \Big\{
   c\big( a,\epsilon ; z, \mu \big) ,a'\big( a,\epsilon ; z, \mu \big),
   r\big( z,\mu \big), w\big( z,\mu \big),
   \Psi (\mu,z)
   \Big\} 
\]
\begin{enumerate}
   %%
   \item ( Household Optimization ) \\
         Taking price functions $ r(\cdot), w( \cdot ) $ and the law of motion $\Psi(\cdot)$ as given,
         $ \Big\{ c( a,\epsilon; z, \mu ) , a'( a,\epsilon ; z, \mu ) \Big\}$ must satisfy
         \begin{align}
              & c( a, \epsilon ; z,\mu )^{ - \sigma } \ge
            \beta \mathbb{E} \bigg\{
            \Big( 1+ \tilde{r} \Big( z', \Psi( z, \mu ) \Big) \Big) c\Big( a'( a, \epsilon; z, \mu ),
            \epsilon'; z', \Psi( z,\mu ) \Big)^{- \sigma}
            \bigg\} \label{eq:euler} \\
              & c( a, \epsilon ; z,\mu ) + a'( a,\epsilon; z, \mu ) = \big( 1 + \tilde{r}(z, \mu ) \big)
            a + w \big( z, \mu \big) \tau( \epsilon )  \label{eq:budget_constraint}
         \end{align}
         with the euler holding with equality if $ a'\big( a, \epsilon; z , \mu \big) > \underline{a} $.

         %%
   \item ( Firm Optimization ) \\
         Given price function $ r( \cdot ) , w ( \cdot ) $, firms choose capital $K$ and labor $L$ according to
         \begin{align}
            r( z, \mu ) & = \alpha e^{z} K^{ \alpha-1 } L^{ 1- \alpha } - \delta \\
            w( z, \mu ) & = ( 1- \alpha ) e^{z} K^{ \alpha } L^{ - \alpha }
         \end{align}

         %%
   \item ( Market Clearing ) \\~\\
         Labor market clears:
         \begin{equation}
            \label{eq:labor_mkt}
            L = \int \epsilon \mu( da, d\epsilon )
         \end{equation}
         %%%%%
         Asset market clears:
         \begin{equation}
            \label{eq:asset_mkt}
            K = \int a \mu( da, d \epsilon )
         \end{equation}
         %%%%%
         Good market clear:
         \begin{equation}
            \label{eq:asset_mkt}
            \int c( a, \epsilon ; z, \mu) \mu( da, d \epsilon ) + \int a'( a,\epsilon ; z, \mu) \mu( da, d \epsilon )
            = \exp{( z )} F( K,L ) + ( 1- \delta ) K
         \end{equation}
         %%
   \item Government budget constraint is satisfied

         %%
   \item ( Evolution of distribution ) \\
         For any $ \epsilon' \in E $ and measurable set $ \mathcal{A} \subseteq A $, next period distribution implied by the
         law of motion $ \Psi $ is consistent with the households' policies
         \begin{equation}
            \label{eq:law_of_motion}
            \Psi( z, \mu ) \Big( \mathcal{A} \times \{\epsilon'\} \Big) = \int
            %
            \ \mathbf{1} \Big\{  a'( a, \epsilon; z, \mu ) \in \mathcal{A} \Big\} \ \pi( \epsilon' | \epsilon ) \ \mu( da, d \epsilon )
            %
         \end{equation}
\end{enumerate}

% SECTION MODEL (END)
%..........................................................................................................................

\newpage
%**************************************************************************************************************************
%%%%%%% METHOD
%--------------------------------------------------------------------------------------------------------------------------
\section{Method} % (fold)
\label{sec:method}

At a broad level, the solution method includes three steps
\begin{enumerate}
   %%
   \item Approximate the infinite dimensional equilibrium objects, which in our example involves the
         policy functions of the household and cross-sectional distribution of individual states, by some
         finite dimensional object. This allows us to have a finite parametrization of the model that we
         shall refer to as the \emph{discrete model}.

         %%
   \item Compute the stationary equilibrium of the \emph{discrete} model. This will take into account
         the idiosyncratic uncertainty but \textbf{not} the uncertainty coming from aggregate shocks.

         %%
   \item Linearize the discrete model equations with respect to our finite parametrization around
         steady-state and compute the rational expectation solution of the linearized system taking into
         account aggregate shocks.
\end{enumerate}

%%%%%%% Finite-Dimensional Approximation
%--------------------------------------------------------------------------------------------------------------------------
\subsection{Finite-Dimensional Approximation} % (fold)
\label{sub:finite_dimensional_approximation}
Looking at the equilibrium definition we see that there are two infinite-dimensional objects that need
to be approximated if we intend to solve the model using \citet{reiter} method.

First, we need a finite representation of household's individual state distribution. I will offer two
options here, one that keeps track of histogram defined on finite number of points of the state space and
another that tries to approximate the distribution by a density function of a finite dimensional family.
Second, household's decision rules characterized by \eqref{eq:euler} are the solution of an functional equation and
must also be approximated.

%..........................................................................................................................
\textbf{Distribution I} \
The first option comes from \citet{reiter} and involves approximating the distribution of households by a
finite number of mass points on a predefined grid for $ \mathcal{A} \defeq \big\{ a_j \big\}_{j=1}^{N_a} $ and productivity $E$.
%%
Let $ \Phi( a_j, \epsilon ) $ denote the fraction of households with productivity level $ \epsilon $ and
asset holdings $ a_j $. The evolution of this distribution is implied by the decision rules as follows.
For any pair $ ( a_{j'}, \epsilon' ) \in \mathcal{A}\times E $, next period mass at the point is given by
%%
\begin{equation}
   \Phi' ( a_{j'}, \epsilon' ) = \sum_{ \epsilon \in E } \pi( \epsilon, \epsilon' )
   \sum_{ j=1 }^{N_a}
   \omega_{j,\epsilon, j'}
   \times  \Phi(a_j, \epsilon) \\
\end{equation}
where $\omega_{j,\epsilon, j'}$ is a function of savings of the household
\begin{equation*}
   \label{eq:weight}
   \omega_{j,\epsilon, j'} = \begin{cases}
   \frac{ a' - a_{j'-1} }{ a_{j'} - a_{j'-1} } & \text{ if }
   a'( a_j, \epsilon ; z, \mu ) \in \Big[ a_{j'-1}, a_{j'} \Big] \\ \ & \ \\
   \frac{ a_{j'+1} - a' }{ a_{j'+1} - a_{j'} } & \text{ if }
   a'( a_j, \epsilon ; z , \mu) \in \Big[ a_{j'}, a_{j'+1} \Big] \\ \ & \ \\
   0 & \text{o.w.}
   \end{cases}
\end{equation*}

\todo[inline]{Here's something I might need to add.}
% Note that I have already replaced the true aggregate state $ ( z,\mu ) $ with the approximate aggregate state
% $ ( z, \Phi ) $ in the decision rule of the household.

\textbf{Distribution II} \
The second alternative follows \citet{winberry} and tries to approximate the distribution over assets
conditional on a realization of labor productivity $ \epsilon $ - $\mu( da,\epsilon )  $ -
using the following parametric family
%%%
\begin{equation}
   \label{eq:density}
   g_{\epsilon} \Big( a ; g_{\epsilon}^0, \big\{ g^i_{\epsilon} ,m^i_{\epsilon}\big\}_{i=1}^{n_g} \Big)
   = g_{\epsilon}^{0} \exp \left\{
   g_{\epsilon}^1( a - m^1_{\epsilon} ) +
   \sum_{i=1}^{n_g} g^1_{\epsilon} \bigg[ ( a - m^1_{\epsilon} )^i - m^i_{\epsilon}\bigg]
   \right\}
\end{equation}
%%%
where $ n_g $ denotes the order of approximation, $ \big\{ g^i_{\epsilon} \big\}_{i=0}^{n_g}$ are the parameters of the distribution
which must be consistent with the centralized moments $ \mathbf{m}_{\epsilon} \defeq \big\{ m^i_{\epsilon}\big\}_{i=1}^{n_g} $, that is,
%%%
\begin{equation}
   \label{eq:consitency}
   %%%
   \begin{split}
      m^1_{\epsilon} & = \int a g_{\epsilon}( a ) da                                  \\
      m^i_{\epsilon} & = \int \big( a-m^1_{\epsilon} \big)^i g_{\epsilon}( a )da 
   \end{split}
\end{equation}
%%%

Turns out that finding a solution of this non-linear system of equation is hard and without
an good initial guess convergence is uncertain. To solve this issue we follow the suggestion by \citet{algan} and find the coefficients - except for $ g^0_{\epsilon} $ - as the solution to the following minimization problem
\begin{equation*}
   \label{eq:min_denscoeff}
   \min_{ \quad \big\{ \rho_{ \epsilon}^i\big\}_{i=1}^{n_g} } \quad \int g \Big( a; 1, \big\{\rho_{ \epsilon }^i ,m^i_{\epsilon}\big\}_{i=1}^{n_g} \Big) da
\end{equation*}
To see that, note that the first order conditions of the problem correspond exactly to consistency conditions
\eqref{eq:consitency}. Turns out that solving this minimization problem is way easier than solving the system
\eqref{eq:consitency} and it works even without a good initial condition.

This approximation reduces the infinite-dimensional distribution $\mu$ to a set of moments
$ \big\{\mathbf{m}_{\epsilon}\big\}_{\epsilon \in E} $. But note that we still need to derive the law of motion
in terms of this approximate aggregate state. Although current density together with decision rules pin down the
distribution for next period, the distribution has no reason to be an element of the parametric family
\eqref{eq:density}.
That in principle presents us with a problem. But note that our reduced state only requires us to keep track of
next period moments, from which we can derive next period approximated density through consistency conditions.
Next period moments in turn are determined as
%
\begin{align*}
   {m_{\epsilon}^1}' & = \sum_{ \tilde{\epsilon} \in E }
   \frac{ \pi ( \tilde{\epsilon} ) \pi( \tilde{\epsilon}, \epsilon ) }{ \pi( \epsilon ) }
   \int a'\Big( a, \tilde{\epsilon}; z, \mathbf{m} \Big) g_{ \tilde{\epsilon} }(a) da \\
   %%%
   {m_{\epsilon}^i}' & =   \sum_{ \tilde{\epsilon} \in E }
   \frac{ \pi ( \tilde{\epsilon} ) \pi( \tilde{\epsilon}, \epsilon ) }{ \pi( \epsilon ) }
   \int \Big( a'\Big( a, \tilde{\epsilon}; z, \mathbf{m} \Big)  - {m_{\epsilon}^1}' \Big)^i g_{ \tilde{\epsilon} }(a) da
\end{align*}
%
In practice, I compute the integrals using a Gauss-Legendre quadrature, which specifies nodes
$\big\{ a_j \big\}_{j=1}^{n_q}$ and weights $\big\{ \omega_j \big\}_{j=1}^{n_q}$ and replaces the integrals
with finite sums
\begin{equation*}
   \label{eq:aaa}
   \int a'( a, \tilde{\epsilon}; z, \mathbf{m} ) g_{ \tilde{\epsilon} }(a) da \approx
   \sum_{j=1}^{n_q} \omega_j a'( a_j, \tilde{\epsilon}; z, \mathbf{m} ) g_{ \tilde{\epsilon} }(a_j)
\end{equation*}
%..........................................................................................................................

\textbf{Household Decision Rules}

Optimality conditions in terms of the new (finite-dimensional) aggregate state are given by
\begin{align*}
     & c \Big( a, \epsilon; z, \Phi \Big)^{ - \sigma } \ge \beta \mathbb{E}_{z'|z}
   \Bigg\{
   \big( 1+\tilde{r}( z', \Phi')  \big)
   \sum_{ \epsilon' \in E } \pi( \epsilon'| \epsilon )  c \Big( a'( a, \epsilon; z, \Phi ), \epsilon'; z', \Phi' \Big)^{- \sigma}
   \Bigg\} \\
     & c\big( a, \epsilon; z, \Phi \big) + a'\big( a, \epsilon; z, \Phi \big) =
   \big( 1 + r( z, \Phi ) \big) a + w ( z,\Phi ) \tau( \epsilon )
\end{align*}
where conditional expectation of future marginal utility has been broken into two pieces: (i) the expectation
with respect to idiosyncratic shocks is taken explicitly, (ii) expectation with respect to aggregate shocks is only
implicitly in the expectation operator and will be taken into account only during the perturbation step.

Household's savings behavior is characterized by a critical level of assets $ \chi_{ \epsilon } $
at which the borrowing constraint starts binding and a smooth function for $ a > \chi_{ \epsilon } $.
I follow \citet{reiter} and approximate the savings rule by a piecewise linear spline with knot points
$ a_{ i, \epsilon}  = \chi_{ \epsilon } + x_i , \ i = 1, \ldots, n_s$ with
$ 0 = x_1 < \ldots < x_{n_s} $.

For a given aggregate state $ ( z, \Phi ) $, household's savings policy is then represented by
$n_{\epsilon} \times n_s$ coefficients giving for each idiosyncratic shock $ \epsilon $ the critical
level $ \chi_{\epsilon} $ and savings at $ a_{2, \epsilon}, \ldots, a_{n_s, \epsilon} $
- note that savings at $ a_{1,\epsilon} $ equals $\underline{a}$ by construction.

These coefficients are collected into the vector $ \boldsymbol{\theta}( z, \Phi ) $ and the approximated
consumption function is then written as $ \hat{c} \big( a, \epsilon ;\  \bsy{\theta}(z, \Phi) \ \big) $.
%%%
Given this approximation choice, household's optimality conditions are approximated using
collocation, which forces \eqref{eq:euler}-\eqref{eq:budget_constraint} to hold exactly on the set of nodes
$ \big\{ a_{i, \epsilon }, \epsilon \big\}_{i\in\{1,\ldots, n_s\},\epsilon\in E}$
\begin{align*}
     & \hat{c} \Big( a_{i,\epsilon}, \epsilon; \ \bsy{\theta}( z,\Phi ) \ \Big)^{ - \sigma } = \beta \mathbb{E}_{z'|z}
   \Bigg\{
   \big( 1 + \tilde{r}( z', \Phi')  \big)
   \sum_{ \epsilon' \in E } \ \pi( \epsilon'| \epsilon ) \ \hat{c} \Big( \hat{a}'( a_{i,\epsilon}, \epsilon; \ \bsy{\theta}( z,\Phi ) \ ),
   \epsilon' ; \ \bsy{\theta}( z', \Phi' ) \ \Big)^{- \sigma}
   \Bigg\} \\
     & \hat{c} \Big( a_{i,\epsilon}, \epsilon; \ \bsy{\theta}( z,\Phi ) \ \Big) +
   \hat{a}'\Big( a_{i,\epsilon}, \epsilon; \ \bsy{\theta}( z,\Phi ) \ \Big) =
   \big( 1 + \tilde{r}( z, \Phi ) \big) a_{i,\epsilon} + w( z, \Phi ) \tau( \epsilon )
\end{align*}

The notation is intended to make clear how differently we treat the dependence of policy on
idiosyncratic and aggregate states. Policy dependence on individual states is explicitly consider and
parametrized by finite-dimensional vector $  \bsy{\theta}(z, \Phi)  $, whose value at steady state
$\bsy{\theta}^*$ is determined by our collocation scheme. Policy depends on aggregates states only
insofar as that parametrization varies with $ (z, \Phi) $, and that dependence is solved
at the perturbation step only.

\begin{quote}
   Mkcay and Reis: We approximate the policy rules for savings by piece-wise linear splines with
   100 knot point each. We deal with the borrowing constraint in the approximation of the policy
   functions by parameterizing the point at which the borrowing constraint starts binding
\end{quote}

\textbf{Approximate Equilibrium Conditions} \\ Given all these approximations, our definition of the
\textbf{recursive competitive equilibrium} is reduced to a finite set of equations.
To collect these conditions using a compact notation define the state vector $ \mathbf{x} = (\Phi,z, K ) $ which contains
$2 + \big(N_a \times n_{\epsilon} \big)$
predetermined variables
and the control vector $ \mathbf{y} = ( \bsy{\theta}, r, w )  $ of size
$2 +\big(n_s \times n_{\epsilon}\big)$.
Therefore, the our discrete equilibrium can be resumed by a system of nonlinear equations
\begin{equation}
   \label{eq:SYSTEM}
   \mathbb{E}_{z'|z} \Big[ f(\mathbf{y}',\mathbf{y},\mathbf{x}',\mathbf{x}, \omega') \Big] = 0
\end{equation}

where $f$ is given by in the case we choose to represent the distribution by histogram (\textbf{Distribution I})
\begin{empheq}[left=\empheqlbrace]{gather*}
   %% transition
   \Phi' ( a_{j'}, \epsilon' ) - \sum_{ \epsilon \in E } \pi( \epsilon, \epsilon' )
   \sum_{ a_j \in \mathcal{A} }
   \Big( \omega_{j,\epsilon, j'} \Phi(a_j, \epsilon) \Big) \quad a_{j'}\in \mathcal{A}, \epsilon \in E
   \\ \ \\
   %% Exogenous
   z'  - \rho_z z - \sigma_{z} \omega_z' \\ \ \\
   % ( \tau^k )'  - \rho_k \tau^k - \sigma_{k} \omega_k' \\ \ \\
   %
   %% Consistency
   K' \ - \sum_{\epsilon \in E}\sum_{ j=1 }^{N_a} a_j \Phi'( a_j, \epsilon ) \\ \ \\
   %
   %% Government Budget Constraint
   % T \bar{L} - \tau^k  K \\ \ \\
   %
   %% Prices
   r - \alpha e^{z} K^{ \alpha-1 } L^{ 1- \alpha } + \delta \\
   w - ( 1- \alpha ) e^{z} K^{ \alpha } L^{ - \alpha }   \\ \ \\
   %
   %% Euler eq
   \hat{c} \Big( a_{i,\epsilon}, \epsilon; \ \bsy{\theta} \ \Big)^{ - \sigma } = \beta
   \Bigg\{
   \big( 1 + r'\big)
   \sum_{ \epsilon' \in E } \ \pi( \epsilon'| \epsilon ) \ \hat{c} \Big( \hat{a}'( a_{i,\epsilon}, \epsilon; \bsy{\theta}  ),
   \epsilon' ; \ \bsy{\theta}' \ \Big)^{- \sigma}
   \Bigg\} \\
   %
   \qquad \hat{c} ( a_{i,\epsilon}, \epsilon; \ \bsy{\theta} \ ) + \hat{a}'( a_{i,\epsilon}, \epsilon; \bsy{\theta}  ) =
   ( 1+r ) a_{i, \epsilon} + w \tau ( \epsilon ) ,
   \qquad i =1, \ldots, n_s \text{ and } \epsilon \in E \\
\end{empheq}
{\color{RubineRed} \textsc{Important}:} It is crucial to have the consistency condition in terms of next period assets.
Problems with stable/unstable roots otherwise. 
% subsection finite_dimensional_approximation (end)
%..........................................................................................................................

%%%%%%% Stationary Equilibrium
%--------------------------------------------------------------------------------------------------------------------------
\subsection{Stationary Equilibrium} % (fold)cd
\label{sub:stationary_equilibrium}

In terms of the notation presented above, a \emph{stationary equilibrium} of the \emph{discrete} model
are values $\mathbf{x}^*, \mathbf{y}^*$ such that $f(\mathbf{y}^*,\mathbf{y}^*,\mathbf{x}^*,\mathbf{x}^*, 0)=0$.
Note that this is just a nonlinear system with as many equations ans unknowns - $6 + \big( (N_a+n_s) \times n_{\epsilon}\big)$.
However, the system is too large to be solved with usual numerical algorithms and we need a more stable scheme to search for the equilibrium.

The following algorithm seems to work in practice. Set up an initial guess for capital $K^0$
\begin{enumerate}

   %%%
   \item Back out market-clearing factor prices $r^0 = \alpha K^{ \alpha -1 } L^{1-\alpha} - \delta $ and
         $ w^0 = (1-\alpha) K^{\alpha} L^{-\alpha}$

         %%%
   \item Given fixed prices $ \big( r^0, w^0 \big) $, we can solve the dynamic problem of the agent for
         $\bsy{\theta}^*_0$

         %%%
   \item Using implied decisions, solve for \emph{invariant distribution}, i.e.
         \begin{equation*}
            \Phi^0 ( a_{j'}, \epsilon' ) - \sum_{ \epsilon \in E } \pi( \epsilon, \epsilon' )
            \sum_{ a_j \in \mathcal{A} }
            \Big( \omega_{j,\epsilon, j'} \big( \bsy{\theta}_0^{*} \big) \times \Phi^0(a_j, \epsilon) \Big)
         \end{equation*}

         %%%
   \item Compute the aggregate supply of capital deduced the invariant distribution
         \[
            A^0 = \sum_{\epsilon \in E} \ \sum_{ j=1 }^{N_a} \ a_j \ \Phi^0 ( a_j, \epsilon )
         \]

         %%%
   \item Check $ \lvert A^0 - K^0 \rvert < \epsilon $. If not update the guess for capital $ K^1 $.%
         \footnote{The updating rule may come from a bisection or newton algorithm.}
\end{enumerate}
% subsection stationary_equilibrium (end)
%..........................................................................................................................


%%%%%%% Linearization
%--------------------------------------------------------------------------------------------------------------------------
\subsection{Linearization using Automatic Differentiation} % (fold)
\label{sub:linearization_using_automatic_differentiation}

As is stated on \emph{wiki}

\begin{quote}
   \em
   Automatic differentiation (AD) is a set of techniques to numerically evaluate the derivative of a function specified by a computer program. AD exploits the fact that every computer program, no matter how complicated, executes a sequence of elementary arithmetic operations (addition, subtraction, multiplication, division, etc.) and elementary functions (exp, log, sin, cos, etc.). By applying the \textbf{chain rule} repeatedly to these operations, derivatives of arbitrary order can be computed automatically, accurately to working precision, and using at most a small constant factor more arithmetic operations than the original program.
\end{quote}

The key idea of AD in to use basic derivative rules from calculus, such as the chain rule, in a numerical environment.
The derivatives are then computed together with the evaluation steps and combined with other derivatives using these rules.

As an example, consider the function $f$ defined by $x \rightarrow x \sin x^2$. When evaluating this function at
$x=3$ you computer will compute the sequence of values on the left of table

\begin{table}[H]
   \centering
   \begin{tabu} to 0.8\textwidth { | X[c] | X[c] | } \hline
      $x = 3$           &  $\dot{x} = 1$                       \\ \hline
      $w_1 = x^2$       &  $\dot{w}_1 = 2 x \dot{x}$           \\ \hline
      $w_2 = \sin(w_1)$ &  $\dot{w}_2 = \cos (w_1) \dot{w}_1$  \\ \hline
      $w_3 = x w_2$     &  $\dot{x}w_2 + x \dot{w}_2$          \\ \hline
   \end{tabu}
\end{table}

But for each computation in the left, we can compute the derivatives that aggregate up to the derivative of
the function at last line. What AD does is make derivative computations to happen automatically
when evaluating or interpreting the values on the left side.

These ideas are implemented in \textsc{Matlab} using object-oriented programming (OOP) features to define a new class of value-and-derivative objects. Methods are then associated
to such objects in order to implement standard calculus derivative rules with built-in chain rule.
In particular, methods overload the definitions of standard operations and functions, such as $*$ and $\sin$,
to compute and return not only the value of the expression but also its derivative.

To see a simple example, to compute the value and derivative of our function $f$ at 3, we create a
value-and-derivative object \texttt{x} with attributes $v =3$ (value) and $d=1$ (derivative). When we evaluate
the expression \texttt{x * sin( x\^{}2 )} both the left and right hand side of the table are executed and the return value
is also a value-and-derivative object with attributes $v=1.2363$ and $d = −15.9883$.

\textbf{Julia Implementation} \\
Using AD is made easy on \emph{Julia} environment due to the ForwardDiff package which offers an implementation
of forward mode automatic differentiation (AD) in Julia. The package is built upon the \texttt{Dual} type

\begin{minted}{julia}
   type DualNumber{N, T<:Real} <: Real
   value::T
   partials::Partials{N,t}
   end
\end{minted}
where \texttt{Partials} is defined
\begin{minted}{julia}
   type Partials{N,T}
   values::NTuple{N,T}
   end
\end{minted}

The idea here is to enhance any real number $x$ by an additional term $ x' \epsilon $, where $ x'\in \mathbb{R} $ and $ \epsilon $ is an
\emph{infinitesimal} with the property that $ \epsilon^2 =0 $.
All arithmetic operators must then be extended for the augmented algebra.
This new arithmetic will consist of ordered pairs, elements written $ ( x,x') $,
with ordinary arithmetics on the first component, and first order differentiation arithmetic on
the second component.

Defining a new arithmetic to our \texttt{Dual} types is natural in Julia because of its \emph{multiple dispatch} approach.
We just need to add a new method to deal with \texttt{Dual} numbers on all elementary numerical functions
that evaluates both the original function and the derivative of the function. For example
Base.sin should be handled as
\begin{minted}{julia}
   Base.sin(d::Dual) = Dual(sin(value(d)), cos(value(d)) * partials(d))
\end{minted}

ForwardDiff defines the \texttt{Dual} type and does all the overloading so that we don't need to
worry about it when computing derivatives.
Since our function $ f$, no matter how complicated, ends up being composed entirely of these elementary functions, then
the chain rule enables our derivatives to compose as well. Thus we can differentiate $f$
by passing in a Dual number and looking at the output.

ADD simple example on the jupyter \ldots
% subsection linearization_using_automatic_differentiation (end)
%..........................................................................................................................

%%%%%%% Solving Linearized System
%--------------------------------------------------------------------------------------------------------------------------
\subsection{Solving Linearized System} % (fold)
\label{sub:solving_linearized_system}


\paragraph{Klein method.}
Linearizing \eqref{eq:SYSTEM} around the steady-state yields a first-order linear expectational difference equation
system of the form
%%%
\begin{equation}
   \label{eq:linearized_equil_conditions}
   f_{ \mathbf{y}' } E_t \Big[ \underbrace{ ( \mathbf{y}_{t+1} -  \bar{\mathbf{y}}) }_{ \tilde{\mathbf{y}}_{t+1} } \Big] 
   + f_{ \mathbf{y} }   \tilde{\mathbf{y}}_t  
   + f_{ \mathbf{x}' }  E_t \Big[  \tilde{\mathbf{x}}_{t+1} \Big]
   + f_{ \mathbf{x} }   \tilde{\mathbf{x}}_{t}
   = 0
\end{equation}
%%%
which can be set on Klein's form letting $ \Gamma_0 \defeq \big[f_{ \mathbf{x}' } \quad f_{ \mathbf{y}' } \big] $  and
$ \Gamma_1 \defeq - \big[f_{ \mathbf{x} } \quad  f_{ \mathbf{y} } \big] $ so that
%%%
\begin{equation}
   \label{eq:klein}
   \Gamma_0
   E_t \left\{ 
         \begin{bmatrix}
            \mathbf{x}_{t+1} \\ \mathbf{y}_{t+1}
         \end{bmatrix}
      \right\} = 
   \Gamma_1
   \begin{bmatrix}
      \mathbf{x}_{t} \\ \mathbf{y}_t 
   \end{bmatrix}
\end{equation}
%%%

\paragraph{Sims method.}
Let $X_t \defeq ( \tilde{\mathbf{x}}; \ \tilde{\mathbf{y}} )$. To put the model into Sims' format, we need to get rid of
the expectational terms in
\begin{equation*}
   \Gamma_0 \mathbb{E}_t \Big[ X_{t+1} \Big] = \Gamma_1 X_t
\end{equation*}
where $ \Gamma_0, \Gamma_1$ are defined as before. To do so we substitute $ E_{t} \big\{X_{t+1}\big\} $ by the
combination of its \emph{ex post} realizations plus the appropriate forecast errors.\footnote{%
   For each $y_{i,t+1}$, whenever we have $ E_{t} \big\{ y_{i,t+1} \big\}  $  we substitute to $ y_{i,t+1} - \eta_{i,t+1}$, 
   where $\eta_{i,t+1} $  are \textbf{endogenous} expectational errors determined as part of the solution.
   Endogenous states are all pre-determined, so $ E_t\{x_{1,t+1}\} = x_{t+1} $, while exogenous states have exogenous
   forecast error $ \omega_{t+1} $, so we can write $ E_t \{\mathbf{x}_{2,t+1} \} = \mathbf{x}_{2,t+1} - \omega_{t+1}$
   }
Once we do that, we get to Sims' canonical form
\begin{equation}
   \label{eq:gensys}
   \Gamma_0 X_{t+1} = \Gamma_1 X_{t} +  \Psi \omega_{t+1} + \Pi \eta_{t+1}
\end{equation}
which is on Sims' canonical form and can be solved using Sim's \texttt{gensys} algorithm\footnote{
I actually use a modified version of the gensys algorithm that treats the existence and uniqueness more carefully. The code accompanies the
\texttt{Reiter.jl}.\}.

% subsection solving_linearized_system (end)
%..........................................................................................................................

% SECTION METHOD (END)
%..........................................................................................................................


% ============================================================================================================
%------------------------------------------------ REFERENCE ------------------------------------------------
% ============================================================================================================
\clearpage
\newpage
%%%%% MAC
% \bibliographystyle{/Users/Felipe/Dropbox/TexFolder/plainnat2}
% \bibliography{/Users/Felipe/Dropbox/TexFolder/2nd_year_paper}
%%%%% PC
\bibliographystyle{C:/Users/falves/Dropbox/TexFolder/plainnat2}
\bibliography{C:/Users/falves/Dropbox/TexFolder/reiter_project}

\end{document}
%%%%%%%%%%%%%%%%%%%%%%%%%%%%%%%%%%%%%%%%%%%%%%%%%%%%%%%%%%%%%%%%%%%%%%%%%%%%%%%%%%%%%%%%%%%%%%%%%%%
%%%%%%%%%%%%%%%%%%%%%%%%%%%%%%%%%%%%%%%%%%%%%%%%%%%%%%%%%%%%%%%%%%%%%%%%%%%%%%%%%%%%%%%%%%%%%%%%%%%
%%% End document
